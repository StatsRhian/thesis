%---------------- COMMENT FOR IMPORTING ----------------------
%\RequirePackage{lineno}				
%\documentclass[12pt]{report}		
%\pagestyle{headings}
%\input{ST-input2}							

%\setcounter{chapter}{0}
%\begin{document}								
%\setpagewiselinenumbers
%\linenumbers
%\tableofcontents
%-------------------------------------------------------------
% RESTRICTION: Not exceeding 300 words.

\chapter*{Abstract}
\addcontentsline{toc}{chapter}{Abstract}

Data streams provide a challenging environment for statistical analysis. Data points can arrive at a high velocity and may need to be deleted once they have been observed. Due to these restrictions, standard techniques may not be applicable to the data streaming scenario. This leads to the need for data summaries to represent the data stream. This thesis explores how data summaries can be used to perform clustering and classification on data streams across a broad range of applications.

Spectral clustering is one such technique which prior to this work has not been applicable to the data streaming setting due to the high computation involved. CluStream is an existing method which uses micro-clusters to summarise data streams. We present two algorithms which utilise these micro-cluster summaries to enable spectral clustering to be performed on data streams. The methods were tested on simulated data streams, as well as textured images and hand-written digits.

Distributed acoustic sensing is used to monitor oil flow at various depths throughout an oil well. Vibrations are recorded at very high resolutions, up to 10000 observations a second at each depth. Unfortunately, corruption can occur in the signal and engineers need to know where corruption occurs. We develop a method which treats the multiple time series as a high-dimensional clustering problem and uses the cluster labels to identify changes within the signal.

The final piece of work concerns identifying areas of activity within a video stream, in particular CCTV footage. It is more efficient if this classification stage is performed on a compressed version of the video stream. In order to reconstruct areas of activity in the original video a recovery algorithm is needed. We present a comparison of the performance of two recovery algorithms and identify an ideal range for the compression ratio. 

 %Computing this on raw video can be computationally challenging but compressive sensing is a method which can efficiently store the video images We need to reconstruct areas of activity in the original image from the compressed video. 



%Chapter 2 specifically considers how spectral clustering performs on data summaries. Chapter 3 is a study on a real data stream. In Chapter 4, when performing background subtraction on a compressed video, we are again using a data summary (the compressed version of the video) to perform an analysis (segmenting the stream) crucially without transmitting the data as a whole (the full video stream) which would be much more computationally challenging.




%---------------- COMMENT FOR IMPORTING ----------------------
%\pagebreak											%Comment for importing
%\bibliographystyle{plainnat}		%Comment for importing
%\bibliography{References}				%Comment for importing
%\end{document}									%Comment for importing
%-------------------------------------------------------------
