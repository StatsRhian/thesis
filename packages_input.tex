%Usepackages

\linespread{2}
%\linespread{1.3}
%Maths mode programs
\usepackage{amsfonts,amsmath,amssymb,mathrsfs,amsthm}
%Page format									
\usepackage[top=25mm, bottom=25mm, left=25mm, right=25mm]{geometry}	%left = 38mm	
%Figure programs
\usepackage{graphicx}	
%For tables
% Please add the following required packages to your document preamble:
 \usepackage{booktabs}
 \newcommand{\bftab}{\fontseries{b}\selectfont}
 \usepackage{multirow}
%\usepackage[textsize = tiny]{todonotes}									
%\usepackage[nodayofweek]{datetime}			
\usepackage[authoryear,round]{natbib}
\usepackage{algorithm}
\usepackage{algorithmic}
\renewcommand{\algorithmicrequire}{\textbf{Input:}}
\renewcommand{\algorithmicensure}{\textbf{Output:}}
%\usepackage{algpseudocode}
%\usepackage{tkz-graph}
%\usetikzlibrary{arrows}
\usepackage{xcolor,colortbl}
\usepackage{caption}
\usepackage{subcaption}
%\usepackage{subfig}
\usepackage{tabularx}
%\usepackage{cite}
\newtheorem{mydef}{Definition}
\usepackage{url}
\usepackage{dsfont}
\usepackage{longtable,booktabs}
%\usepackage{fancyhdr}

\newenvironment{example}[1][Example]{\begin{trivlist}
\item[\hskip \labelsep {\bfseries #1}]}{\end{trivlist}}

\usepackage{array}
\usepackage{tikz}
\usetikzlibrary{decorations.pathreplacing}
\newcommand{\tikzmark}[1]{\tikz[overlay,remember picture,baseline=(#1.base)]
  \node (#1) {\strut};}

%\let\MakeUppercase\relax % Stop page headers being capitalised

\newcommand\undermat[2]{%
  \makebox[0pt][1]{$\smash{\underbrace{\phantom{%
          \begin{matrix}#2\end{matrix}}}_{\text{$#1$}}}$}#2}

\DeclareMathOperator*{\argmin}{arg\,min}
\DeclareMathOperator*{\argmax}{arg\,max}
 
\def\Var{{\rm Var}\,}
\def\var{{\rm var}\,}
\def\E{{\rm E}\,}


%macros
\let\oldsection\section
\renewcommand{\section}[1]{
	\setcounter{figure}{0}
	\setcounter{table}{0}
	\setcounter{equation}{0}
	\oldsection{#1}
}

%Ordinal Superscripts
\newcommand{\st}{\textsuperscript{st}~}	%st superscript
\newcommand{\nd}{\textsuperscript{nd}~}	%nd superscript
\newcommand{\rd}{\textsuperscript{rd}~}	%rd superscript
\newcommand{\Th}{\textsuperscript{th}~}	%th superscript

%%Maths Environments
\newenvironment{mat}{\left[ \begin{array}}{\end{array} \right]}
\newenvironment{deter}{\left| \begin{array}}{\end{array} \right|}
\newenvironment{ifbrace}{\left\{ \begin{array}}{\end{array} \right.}

%Table, Equation and Figure Labels
\renewcommand{\thefigure}{\thesection.\arabic{figure}}
\renewcommand{\thetable}{\thesection.\arabic{table}}
\renewcommand{\theequation}{\thesection.\arabic{equation}}

%%Number type
\newcommand{\NN}{\mathbb{N}}	%Natural
\newcommand{\ZZ}{\mathbb{Z}}	%Integer
\newcommand{\RR}{\mathbb{R}}	%Real

%%Statistical Notation
\newcommand{\PP}[1]{\mathbb{P}\left(#1\right)}		%P()
\newcommand{\EE}[1]{\mathbb{E}\left[#1\right]}		%E[]
\newcommand{\VAR}[1]{\mbox{Var}\left(#1\right)}		%Var()
\newcommand{\PiDist}[1]{\pi\left(#1\right)}				%pi()
\newcommand{\cov}[2]{\mbox{Cov}\left(#1,#2\right)}	%Covariance
\newcommand{\corr}[2]{\mbox{Corr}\left(#1,#2\right)}	%Correlation
\newcommand{\Like}[1]{L\left(#1\right)}						%Likelihood
\newcommand{\lLike}[1]{\ell\left(#1\right)}				%log-likelihood

%%Distributions
\newcommand{\Gauss}[2]{\mathcal{N}\left(\quad #1,\quad #2 \quad\right)}
\newcommand{\MVGauss}[3]{\mathcal{MVN}_{#1}\left(\quad #2,\quad #3 \quad\right)}
\newcommand{\IGam}[2]{\mathcal{IG}\left(\quad #1,\quad #2 \quad\right)}
\newcommand{\T}[4]{\mathcal{T}_{#1}\left(\quad #2, \quad #3, \quad #4 \quad \right)}
\newcommand{\dT}[5]{\mathcal{T}_{#1}\left(\quad #2; \quad #3, \quad #4, \quad #5 \quad \right)}
\newcommand{\MVT}[4]{\mathcal{T}_{#1}\left(\quad #2, \quad #3, \quad #4 \quad\right)}
\newcommand{\BETA}[2]{\mathcal{B}\mbox{eta}\left(\quad #1, \quad #2 \quad \right)}
\newcommand{\dBETA}[3]{\mathcal{B}\mbox{eta}\left(\quad #1; \quad #2, \quad #3 \quad \right)}

\newcommand{\Bern}[1]{\mathcal{B}\mbox{ern}\left(\quad #1 \quad\right)}
\newcommand{\Pois}[1]{\mathcal{P}\mbox{oisson}\left(\quad #1 \quad\right)}

%%Mathematical symbols
\newcommand{\half}{\dfrac{1}{2}}												%1/2
\newcommand{\third}{\dfrac{1}{3}}												%1/3
\newcommand{\quarter}{\dfrac{1}{4}}											%1/4
\newcommand{\const}{\! {\rm const.}\; }									%constant
\newcommand{\recip}[1]{\frac{1}{#1}}										%reciprocal - 1/x
\newcommand{\set}[1]{\left\{ #1 \right\} }							%{---}
\newcommand{\seq}[1]{\left( #1 \right)}									%(---)
\newcommand{\e}[1]{{\mbox{exp}}\left\{ #1 \right\}}			%exponent (text)
\newcommand{\Gamm}[1]{\Gamma\left( #1 \right)}				  %Gamma function
\newcommand{\Ind}[1]{\mathbb{I}_{\left\{#1\right\}}}		%Indicator Function
\newcommand{\Id}[1]{{I}_{#1}}														%Identity Matrix
\newcommand{\limit}[2]{\stackrel{\lim }{_{ #1 \to #2 }}}%Limit
\newcommand{\maxover}[1]{\stackrel{\max }{_{ #1 }}}			%max
\newcommand{\minover}[1]{\stackrel{\min }{_{ #1 }}}			%min
\newcommand{\OVec}{\mathbf{0}}													%Zero Vector
\newcommand{\IVec}{\mathbf{1}}													%Ones Vector

%%Commenting
\newcommand{\RD}[1]{\textcolor{blue}{RD: #1}}


%Theorem, Lemma, Proposition, Corollary
\newtheorem{definition}{Definition}[section]
\newtheorem{theorem}[definition]{Theorem}
\newtheorem{lemma}[definition]{Lemma}
\newtheorem{proposition}[definition]{Proposition}
\newtheorem{corollary}[definition]{Corollary}

\newtheoremstyle{mystyle}  % follow `plain` defaults but change HEADSPACE.
  {}   				% ABOVESPACE
  {}   				% BELOWSPACE
  {}  				% BODYFONT
  {0pt}       		% INDENT (empty value is the same as 0pt)
  {\bfseries}	% HEADFONT
  {:}        % HEADPUNCT
  {5pt plus 1pt minus 1pt}				  % HEADSPACE. `plain` default: {5pt plus 1pt minus 1pt}
  {}          % CUSTOM-HEAD-SPEC

\theoremstyle{mystyle}% default
\newtheorem*{eg}{Working Example}

\newenvironment{workedexample}[1]{\begin{eg}\textbf{#1}\\}{\hfill $\blacktriangle$ \end{eg}}
