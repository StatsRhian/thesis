
%---------------- COMMENT FOR IMPORTING ----------------------
% \RequirePackage{lineno}					%Comment for importing
% \documentclass[12pt]{report}		%Comment for importing
% \pagestyle{headings}
% %\input{ST-input2}								%Comment for importing
% %Usepackages
\linespread{2}
%\linespread{1.3}
%Maths mode programs
\usepackage{amsfonts,amsmath,amssymb,mathrsfs,amsthm}
%Page format									
\usepackage[top=25mm, bottom=25mm, left=25mm, right=25mm]{geometry}	%left = 38mm	
%Figure programs
\usepackage{graphicx}	
%For tables
% Please add the following required packages to your document preamble:
 \usepackage{booktabs}
 \newcommand{\bftab}{\fontseries{b}\selectfont}
 \usepackage{multirow}
%\usepackage[textsize = tiny]{todonotes}									
%\usepackage[nodayofweek]{datetime}			
\usepackage[authoryear,round]{natbib}
\usepackage{algorithm}
\usepackage{algorithmic}
\renewcommand{\algorithmicrequire}{\textbf{Input:}}
\renewcommand{\algorithmicensure}{\textbf{Output:}}
%\usepackage{algpseudocode}
%\usepackage{tkz-graph}
%\usetikzlibrary{arrows}
\usepackage{xcolor,colortbl}
\usepackage{caption}
\usepackage{subcaption}
\usepackage{subfig}
\usepackage{tabularx}
%\usepackage{cite}
\newtheorem{mydef}{Definition}
\usepackage{url}
\usepackage{dsfont}
\usepackage{longtable,booktabs}
%\usepackage{fancyhdr}

\usepackage{array}
\usepackage{tikz}
\usetikzlibrary{decorations.pathreplacing}
\newcommand{\tikzmark}[1]{\tikz[overlay,remember picture,baseline=(#1.base)]
  \node (#1) {\strut};}

%\let\MakeUppercase\relax % Stop page headers being capitalised

\newcommand\undermat[2]{%
  \makebox[0pt][1]{$\smash{\underbrace{\phantom{%
          \begin{matrix}#2\end{matrix}}}_{\text{$#1$}}}$}#2}

\DeclareMathOperator*{\argmin}{arg\,min}
\DeclareMathOperator*{\argmax}{arg\,max}
 
\def\Var{{\rm Var}\,}
\def\var{{\rm var}\,}
\def\E{{\rm E}\,}


%macros
\let\oldsection\section
\renewcommand{\section}[1]{
	\setcounter{figure}{0}
	\setcounter{table}{0}
	\setcounter{equation}{0}
	\oldsection{#1}
}

%Ordinal Superscripts
\newcommand{\st}{\textsuperscript{st}~}	%st superscript
\newcommand{\nd}{\textsuperscript{nd}~}	%nd superscript
\newcommand{\rd}{\textsuperscript{rd}~}	%rd superscript
\newcommand{\Th}{\textsuperscript{th}~}	%th superscript

%%Maths Environments
\newenvironment{mat}{\left[ \begin{array}}{\end{array} \right]}
\newenvironment{deter}{\left| \begin{array}}{\end{array} \right|}
\newenvironment{ifbrace}{\left\{ \begin{array}}{\end{array} \right.}

%Table, Equation and Figure Labels
\renewcommand{\thefigure}{\thesection.\arabic{figure}}
\renewcommand{\thetable}{\thesection.\arabic{table}}
\renewcommand{\theequation}{\thesection.\arabic{equation}}

%%Number type
\newcommand{\NN}{\mathbb{N}}	%Natural
\newcommand{\ZZ}{\mathbb{Z}}	%Integer
\newcommand{\RR}{\mathbb{R}}	%Real

%%Statistical Notation
\newcommand{\PP}[1]{\mathbb{P}\left(#1\right)}		%P()
\newcommand{\EE}[1]{\mathbb{E}\left[#1\right]}		%E[]
\newcommand{\VAR}[1]{\mbox{Var}\left(#1\right)}		%Var()
\newcommand{\PiDist}[1]{\pi\left(#1\right)}				%pi()
\newcommand{\cov}[2]{\mbox{Cov}\left(#1,#2\right)}	%Covariance
\newcommand{\corr}[2]{\mbox{Corr}\left(#1,#2\right)}	%Correlation
\newcommand{\Like}[1]{L\left(#1\right)}						%Likelihood
\newcommand{\lLike}[1]{\ell\left(#1\right)}				%log-likelihood

%%Distributions
\newcommand{\Gauss}[2]{\mathcal{N}\left(\quad #1,\quad #2 \quad\right)}
\newcommand{\MVGauss}[3]{\mathcal{MVN}_{#1}\left(\quad #2,\quad #3 \quad\right)}
\newcommand{\IGam}[2]{\mathcal{IG}\left(\quad #1,\quad #2 \quad\right)}
\newcommand{\T}[4]{\mathcal{T}_{#1}\left(\quad #2, \quad #3, \quad #4 \quad \right)}
\newcommand{\dT}[5]{\mathcal{T}_{#1}\left(\quad #2; \quad #3, \quad #4, \quad #5 \quad \right)}
\newcommand{\MVT}[4]{\mathcal{T}_{#1}\left(\quad #2, \quad #3, \quad #4 \quad\right)}
\newcommand{\BETA}[2]{\mathcal{B}\mbox{eta}\left(\quad #1, \quad #2 \quad \right)}
\newcommand{\dBETA}[3]{\mathcal{B}\mbox{eta}\left(\quad #1; \quad #2, \quad #3 \quad \right)}

\newcommand{\Bern}[1]{\mathcal{B}\mbox{ern}\left(\quad #1 \quad\right)}
\newcommand{\Pois}[1]{\mathcal{P}\mbox{oisson}\left(\quad #1 \quad\right)}

%%Mathematical symbols
\newcommand{\half}{\dfrac{1}{2}}												%1/2
\newcommand{\third}{\dfrac{1}{3}}												%1/3
\newcommand{\quarter}{\dfrac{1}{4}}											%1/4
\newcommand{\const}{\! {\rm const.}\; }									%constant
\newcommand{\recip}[1]{\frac{1}{#1}}										%reciprocal - 1/x
\newcommand{\set}[1]{\left\{ #1 \right\} }							%{---}
\newcommand{\seq}[1]{\left( #1 \right)}									%(---)
\newcommand{\e}[1]{{\mbox{exp}}\left\{ #1 \right\}}			%exponent (text)
\newcommand{\Gamm}[1]{\Gamma\left( #1 \right)}				  %Gamma function
\newcommand{\Ind}[1]{\mathbb{I}_{\left\{#1\right\}}}		%Indicator Function
\newcommand{\Id}[1]{{I}_{#1}}														%Identity Matrix
\newcommand{\limit}[2]{\stackrel{\lim }{_{ #1 \to #2 }}}%Limit
\newcommand{\maxover}[1]{\stackrel{\max }{_{ #1 }}}			%max
\newcommand{\minover}[1]{\stackrel{\min }{_{ #1 }}}			%min
\newcommand{\OVec}{\mathbf{0}}													%Zero Vector
\newcommand{\IVec}{\mathbf{1}}													%Ones Vector

%%Commenting
\newcommand{\RD}[1]{\textcolor{blue}{RD: #1}}


%Theorem, Lemma, Proposition, Corollary
\newtheorem{definition}{Definition}[section]
\newtheorem{theorem}[definition]{Theorem}
\newtheorem{lemma}[definition]{Lemma}
\newtheorem{proposition}[definition]{Proposition}
\newtheorem{corollary}[definition]{Corollary}

\newtheoremstyle{mystyle}  % follow `plain` defaults but change HEADSPACE.
  {}   				% ABOVESPACE
  {}   				% BELOWSPACE
  {}  				% BODYFONT
  {0pt}       		% INDENT (empty value is the same as 0pt)
  {\bfseries}	% HEADFONT
  {:}        % HEADPUNCT
  {5pt plus 1pt minus 1pt}				  % HEADSPACE. `plain` default: {5pt plus 1pt minus 1pt}
  {}          % CUSTOM-HEAD-SPEC

\theoremstyle{mystyle}% default
\newtheorem*{eg}{Working Example}

\newenvironment{workedexample}[1]{\begin{eg}\textbf{#1}\\}{\hfill $\blacktriangle$ \end{eg}}

% \usepackage[authoryear,round]{natbib}
%\setcounter{chapter}{1}
% \begin{document}								%Comment for importing
% \setpagewiselinenumbers
% \linenumbers
%tableofcontents
\graphicspath{{Chapter2/figures/}} 
% \graphicspath{{figures/}} % comment for importing
%-------------------------------------------------------------

\chapter{Spectral Clustering for Data Streams}
\label{chap:spectral}
\section{Introduction}
\label{sec:datastreams_intro}

%Appl, data streams no clustering
A \textit{data stream} \citep{Gama2010, Silva2013} is data which arrives in an ordered sequence, continuously and is potentially unbounded in length. Examples can be found in many applications, and the number of application areas is ever increasing as ever more aspects of our lives become monitored by technology. Obvious examples include telecommunications, shopping transactions and customer click data. The use of technology to track industrial processes has increased interest in data streams consisting of sensor data such as audio and visual data. Often an analyst will want to identify clusters within the data stream.

%Clustering is a useful tool used to analyse data.
The goal of clustering algorithms is to separate the data into groups or clusters such that data points within a cluster are similar, and data points in different clusters are dissimilar. Many different types of clustering algorithms exist, centroid type methods such as k-means \citep{MacQueen1967, Lloyd1982} and density based algorithms like DB-Scan \citep{Ester1996}. In this Chapter, we restrict our interests to \textit{Spectral Clustering}, a clustering method which uses the eigenvalues of an \textit{affinity matrix} of the data to perform dimension reduction and cluster in the lower dimensional space. Spectral Clustering is popular, offers good empirical performance and can handle tricky, non standard data sets. An introduction to Spectral Clustering is given in Section \ref{sec:sc_background}. 

Although there does not exist an algorithm for performing Spectral Clustering on data streams, there exist advanced Spectral Clustering techniques which are discussed in Section \ref{sec:ad_spec}. These algorithms address some of the issues met in data streaming, by working with large scale data or incremental data sets but do not meet all the challenges of implementing Spectral Clustering in data streams.

A popular algorithm to handle clustering for data sets is the Clustream algorithm \citep{Aggarwal2003}, the details of which are explained in Section \ref{sec:microSpec}. Using Clustream with Spectral Clustering on data streams has not been explored before.  We introduce two Spectral Clustream algorithms,  weighted and unweighted variants and explore their performance on simulated and real data in Section \ref{sec:clustream_exp}. Conclusions and future work are discussed in Section \ref{sec:clustream_conc}.


\section{Spectral Clustering Background}
\label{sec:sc_background}
In this section we motivate Spectral Clustering and introduce the Spectral Clustering algorithm by first noting the link between Spectral Clustering and graph partitioning problems. 

\subsection{Motivation}

As discussed in Section \ref{sec:datastreams_intro} the goal of clustering algorithms is to partition data  $X = \{ x_1, \ldots, x_n \}, x_i \in \mathbb{R}^d$, into $k$ disjoint classes such that each $x_i$ belongs to exactly one class. Data sets can have underlying true clusters of all shapes and sizes, for example, they can be spherical and convex as in Figure \ref{fig:compact} or connected but non-convex as in Figure \ref{fig:connected}.

\begin{figure}[h!]
  \centering

  \begin{subfigure}{0.4\textwidth}
    \centering
    \includegraphics[width = 0.8\textwidth]{my_compact}
    \caption{Convex clusters}
  \label{fig:compact}
  \end{subfigure}
  \begin{subfigure}{0.4\textwidth}
    \centering
    \includegraphics[width = 0.8\textwidth]{my_connected}
     \caption{Non-convex clusters}
     \label{fig:connected}
     \end{subfigure}
  \caption{Examples of different types of clusters}
  \label{fig:compact_connected}
\end{figure}

 Data which is convex may be simple to cluster as the gaps between clusters are easy for simple clustering algorithms like k-means to identify. Connected but non-convex data sets can be much more challenging than convex data sets, and can cause some simple clustering algorithms to fail. The reason that centroid based clustering algorithms such as k-means struggle with data sets like that shown in Figure \ref{fig:connected} is that k-means clusters the data based on how similar they are to cluster centroids. Spectral Clustering instead clusters data based on how \textit{similar} they are to all other data points, which can lead to good quality segmentation on even these difficult cases.  We do not formally address what is meant by similarity here, but will define this fully in Section \ref{sec:affinity}.

The similarity between data points can be neatly represented in a \textit{similarity graph}.  We can then restate the clustering problem as a graph partitioning problem where we wish to find a partition of the graph such that the edges between different groups have low weights (which corresponds to data points being dissimilar) and the edges within a group have high weights (the data points are similar).

In order to introduce Spectral Clustering we first discuss introduce some graph notation and discuss graph cut problems. We will then describe the Spectral Clustering algorithm, and discuss in more detail the notion of similarity. 

\subsection{Graph cut problems} 

Data can be represented as a similarity graph, $G = (V,E)$ where each vertex $v_i \in V$ represents a data point $x_i$. The graph can then be described by an \textit{adjacency matrix}. Adjacency matrices are a way of depicting the graph structure with binary entries denoting which vertexes are connected by edges and which are not. Figure \ref{fig:sim_graphs} depicts two similarity graphs. Their corresponding adjacency matrices are given in equation \eqref{eq:adj_mat}.  A  value of $1$ in cell (2,3) implies that vertexes $v_2$ and $v_3$ are connected by an edge. Note that both of the example adjacency matrices given below are symmetric, which can be expected as we are dealing with undirected graphs.

\begin{figure}[h!]
  \centering
  \begin{subfigure}{0.4\textwidth}
    \centering
    \includegraphics[width = 0.6\textwidth]{adj_square.png}
 %   \caption{cap}
 % \label{fig:adf_square}
  \end{subfigure}
  \begin{subfigure}{0.4\textwidth}
    \centering
    \includegraphics[width = 0.6\textwidth]{adj_tree.png}
%  \caption{Cap}
%  \label{fig:adj_tree}
  \end{subfigure}
  \caption{Two similarity graphs}
  \label{fig:sim_graphs}
\end{figure}

\begin{equation}
\label{eq:adj_mat}
\renewcommand*{\arraystretch}{.5}
  \begin{pmatrix}
    0 & 1 & 0 & 1 \\
    1 & 0 & 1 & 0 \\
    0 & 1 & 0 & 1 \\
    1 & 0 & 1 & 0 
    \end{pmatrix}
\hspace{1.3in}
\renewcommand*{\arraystretch}{.5}
  \begin{pmatrix}
    0 & 1 & 1 & 1 \\
    1 & 0 & 0 & 0 \\
    1 & 0 & 0 & 0 \\
    1 & 0 & 0 & 0 
    \end{pmatrix}
\end{equation}

The \textit{weighted adjacency matrix} (also called an \textit{affinity matrix}) of a similarity graph is the matrix $W = (w_{ij})_{i,j = 1,\ldots, n}$. The weight $w_{ij}$ is the similarity between vertexes $v_i$ and $v_j$.  If $w_{ij}=0$ this means that the vertexes $v_i$ and $v_j$ are not connected by an edge. Again the affinity matrix will be symmetric, that is $w_{ij} = w_{ji}$. 

In order to create a graph partition we need to cut the edges in the graph. Non empty subsets of $V$, $A$ and $B$ will form a partition of the graph $G$ if $A \cap B = \emptyset  $ and $A \cup B = V$.  
The weight of the cut can be calculated by summing the weights of the edges which will be broken when a cut is made. In order to find a good partition of the graph, we wish to choose $A$ and $B$  such that some cut criterion is minimised.  The simplest cut criterion is

\begin{equation}
  \text{cut(A,B)} = \sum_{i \in A, j \in B} w_{ij},\\
  \label{eq:cut}
\end{equation}
where the notation $i \in A$ is short hand to mean the set of indexes $\{ i | v_i \in A \}$.

The Minimum cut \citep{Wu1993} is the cut which minimises equation \eqref{eq:cut}. This is fairly easy to solve \citep{Stoer1997} however the  minimum cut does not always produce a desirable graph partitioning; it tends to create unbalanced partitions, separating one vertex from the rest of the graph. To understand why this happens, note that the number of edges cut in mincut will be  $|A| \times |B|$ which is minimised by the solutions $|A| = 1$ or $|B| = 1$. In order to avoid this, we can specify that the sets $A$ and $B$ are reasonably large in some way. Two common objective functions used to avoid this issue are the RatioCut \citep{Hagen1992} and the normalised cut, Ncut \citep{Malik2000}. 

Both RatioCut and Ncut attempt to normalise the weight of the cut by introducing the size of sets $A$ and $B$. In RatioCut, the size of $A$ is measured by its number of vertexes $|A|$, while in Ncut the size is measured by the weights of its edges $\text{vol}(A) = \sum_{i \in A}d_i$ where $d_i= \sum_{j = 1}^n w_{ij}$ is the \textit{degree} of a vertex $v_i \in V$. The definitions of RatioCut and Ncut are as follows, 

\begin{equation}
  \text{RatioCut(A,B)} = \frac{\text{cut} (A, B)}{|A|} + \frac{\text{cut}(A, B)}{|B|}, 
    \label{eq:ratiocut}
\end{equation}

\begin{equation}
  \text{Ncut(A,B)} = \frac{\text{cut}(A, B)}{\text{vol}(A)} + \frac{\text{cut}(A, B)}{\text{vol}(B)}.
    \label{eq:ncut}
\end{equation}

The main idea in Ncut is that large clusters will increase the denominator vol$(A)$ and thus decrease Ncut$(A,B)$. This will encourage splitting the data into fairly evenly sized clusters, and avoid the minimum cut issue of segmented isolated points. This can be seen in Figure \ref{fig:min_norm_cut}  which depicts both the minimum cut and Ncut solutions for a particular graph. The shaded/non-shaded regions represent the partitioning. The minimum cut isolates one vertex from the rest of the graph, whilst the Ncut provides a more balanced and sensible partition. 

\begin{figure}[h!]
  \centering
  \begin{subfigure}{0.4\textwidth}
    \centering
    \includegraphics[width = \textwidth]{my_min_cut.png}
    \caption{Minimising the cut.}
  \label{fig:min_cut}
  \end{subfigure}
  \begin{subfigure}{0.4\textwidth}
    \centering
    \includegraphics[width = \textwidth]{my_norm_cut.png}
  \caption{Minimising the normalised cut.}
  \label{fig:norm_cut}
  \end{subfigure}
  \caption{Two solutions to the bi-partition problem. The partitioning is indicated by shading/non-shading of nodes.}
  \label{fig:min_norm_cut}
\end{figure}

Although the partitioning has been improved, the previously easy to solve mincut problem has been replaced with minimising the normalised cut which is an NP-hard problem \citep{Wagner1993}. Spectral Clustering is a way to solve relaxed versions of these problems \citep{Luxburg2008} which are much easier to solve.  It has been shown that the two graph cut optimisations given in equations \eqref{eq:ratiocut} and \eqref{eq:ncut} can be formulated in terms of the spectral decomposition of the graph Laplacian matrices given in equations \eqref{eq:laplacian} and \eqref{eq:laplacian_symm} respectively.  

\begin{eqnarray}
\label{eq:laplacian}
 L =& D - W\\
\label{eq:laplacian_symm}
 L_{\text{symm}} =& D^{-1/2}LD^{-1/2} 
\end{eqnarray}

Here $W$ is the affinity matrix of the similarity graph,  The degree matrix $D$ is defined as the diagonal matrix with the degrees $d_1, \ldots d_n$ on the diagonal. The Laplacian $L$ relates to a relaxation of minimising the RatioCut and $L_{\text{symm}}$ relates to the relaxation of the Ncut \citep{chung1997spectral}. There is no guarantee on the quality of the solution of the relaxed problem compared to the exact solution and there do exist pathological cases which are arbitrarily bad. However, several papers which investigate the quality of the clustering of Spectral Clustering \cite{Spielman1996} and \cite{Kannan2004} find Spectral Clustering to provide good solutions. 

The Spectral Clustering algorithm that we use \citep{Malik2000} uses the symmetric Laplacian  $L_{\text{symm}}$. The full Spectral Clustering algorithm is given in Algorithm \ref{alg:njw}. Note that the number of clusters is assumed to be known and this will be the case throughout this chapter.


\begin{algorithm}
\caption{NJW Spectral Clustering algorithm}
\begin{algorithmic}[1]
\REQUIRE Data set $X = \{x_1,\ldots, x_n \}$, number of clusters $k$
\ENSURE $k$-way partition of the input data
\STATE Construct the affinity matrix $W = (w_{ij})_{i,j = 1,\ldots, n}$ %by the following Gaussian kernel function:
%\begin{equation*}
 % w_{i,j} = \exp  \left( - \frac{\| x_i - x_j \|^2}{2 \sigma^2} \right), i,j = 1, \ldots,n.
%\end{equation*}

\STATE Compute the normalised Laplacian matrix  $L_{\text{symm}} = D^{-1/2}( D - W )D^{-1/2}$, where $D$ is the diagonal matrix with $D_{ii}=\sum_{j=1}^{n} w_{ij}.$
\STATE Compute the $k$ eigenvectors of $L_{\text{symm}}$, $v_1, v_2,\ldots , v_k,$ associated with the $k$ smallest eigenvalues, and form the matrix $ V = [v_1,v_2, \ldots ,v_k]$.
\STATE Renormalise each row of $V$ to form a new matrix $Y$.
\STATE Partition the $n$ rows of $Y$ into $k$ clusters using  k-means.
\STATE Assign the original data point $x_i$ to the cluster $l$ if and only if the corresponding row $i$ of the matrix $Y$ is assigned to the cluster $l$.
\end{algorithmic}
\label{alg:njw}
\end{algorithm}

Once the Laplacian has been calculated, one computes the $k$ eigenvectors which correspond to the $k$ smallest eigenvalues of the Laplacian. A matrix $Y \in \mathbb{R}^{n \times k}$ is created, where each column is an eigenvector of the Laplacian, with length $n$. We can view this matrix $Y$ as an embedding of the original data $X$ into a lower dimensional subspace. When represented in this low subspace the clustering problem is often easier, and can be solved with a simple clustering algorithm such as k-means. For example, Figure \ref{fig:spirals_original} shows a data set of three spirals depicted in the original feature space. This is visually quite difficult to cluster. Figure \ref{fig:spirals_embedded} plots the same data set but embedded in the lower dimension, plotting the first eigenvector of the Laplacian against the second eigenvector. Clustering in the embedded space is easy even for k-means to solve. 


\begin{figure}[h!]
  \centering
  \begin{subfigure}{0.4\textwidth}
    \centering
    \includegraphics[width = \textwidth, height = \textwidth]{code_embedding/3_spiral_original.png}
    \caption{Data in the feature space}
  \label{fig:spirals_original}
  \end{subfigure}
  \begin{subfigure}{0.4\textwidth}
    \centering
    \includegraphics[width =\textwidth, height = \textwidth]{code_embedding/3_spiral_embedded.png}
     \caption{Data in the 2d-embedding space}
     \label{fig:spirals_embedded}
     \end{subfigure}
  \caption{Triple spiral data set viewed in feature space and eigenvector embedding}
  \label{fig:triple_spirals}
\end{figure}


After the embedding, k-means can be used to cluster the $n$ rows of $Y$ into $k$ clusters. Finally, assign the cluster label given to each row $Y_i$ to the corresponding original data point $x_i$. Now that we have introduced the general Spectral Clustering algorithm, we will discuss the notion of similarity in more detail.

\subsection{Choice of affinity matrix}
\label{sec:affinity}

One of the key factors of Spectral Clustering is the affinity matrix $W = (w_{ij})_{ i,j = 1, \ldots, n}$  which represents the pairwise similarities between all data points $x_i$ and $x_j$. A popular choice is to use the Gaussian kernel, 

\begin{equation}
  \label{eq:gaussian_affinity}
    w_{i,j} = \exp  \left( - \frac{\| x_i - x_j \|^2}{2 \sigma^2} \right), \; i, j = 1, \ldots, n,
\end{equation}
where the parameter $\sigma$ controls the width of the local neighbourhoods which we want to model. If $x_i$ and $x_j$ are very close, then $w_{ij} \rightarrow 1 $, and if they are far apart $w_{ij} \rightarrow 0$. A Gaussian kernel affinity matrix will have ones along the diagonal and is symmetric $(w_{ij} = w_{ji})$.

The scaling parameter $\sigma$ is usually chosen manually. \cite{Ng2001} automatically choose $\sigma$ by running their clustering algorithm repeatedly for a number of values of $\sigma$ and selecting the one which provides least distorted clusters. \cite{Zelnik-Manor2004} argue that for data which has a cluttered background, or multi-scale data, one global parameter choice for $\sigma$ is not sufficient. They calculate a localised parameter $\sigma_i$ for each data point $x_i$ based on its neighbourhood. Using a localised $\sigma_i$ can deal well with multi-scale data, but requires the user to choose the size of the neighbourhood in order to calculate $\sigma_i$. 

If we mainly wish to model the local relationships then using all of the possible pairwise data similarities may not be necessary. It is possible to use a weighted k-nearest neighbour structure \citep{Luxburg2008} to build the affinity matrix once corrections have been made to ensure that this matrix is symmetric. Another option is to choose some threshold $\epsilon$ and only consider connections between data points whose pairwise similarities are greater  than some threshold $\epsilon$. This is an $\epsilon$-neighbourhood graph as shown in equation \eqref{eq:epsilon_graph}. 

%Although it is possible to weight this graph by $\epsilon$, if we choose $\epsilon$ to generate a small $\epsilon$-neighbourhood, then the differences between the weights will be so small that weighting may become eligible.

\begin{equation}
\label{eq:epsilon_graph}
  w^*_{ij}=\left\{
  \begin{array}{@{}ll@{}}
    1, & \text{if}\ w_{ij} > \epsilon \\
    0, & \text{otherwise}
  \end{array}\right.
\end{equation} 

Using this construction will give a sparse affinity matrix instead of a fully connected graph, which will help lower the computational complexity. 

\section{Advanced Spectral Clustering}
\label{sec:ad_spec}

In the previous section we introduced Spectral Clustering via graph partitioning and discussed options for creating affinity matrices. Our overall aim is to perform Spectral Clustering on data streams.   First, we consider some of the challenges that make clustering in data streams so difficult, and look at the approaches that exist to deal with these challenges in the Spectral Clustering setting.

 The first challenge that is addressed in dealing with big data. One of the main difficulties in data streaming is the pure volume of data available and the methods discussed in Section \ref{sec:big_data} offer methods to perform Spectral Clustering on big data.  Another difficulty that arises in data streaming is the ability to update the current clustering result  as new data arrives.  This problem is called Incremental Spectral Clustering and is discussed in Section \ref{sec:incremental}.

It is stressed that these problems (large data and incremental data) are just sub-problems of what makes data streaming difficult, and therefore none of the methods described below are capable of dealing with the full data streaming problem. The even more challenging setting, data streams, will be addressed in Section \ref{sec:microSpec}.    

\subsection{Large-scale Spectral Clustering}
\label{sec:big_data}

Although Spectral Clustering has been shown to perform well empirically on simple data sets, computational problems arise as the data set size increases.  Spectral Clustering can be challenging for very large data sets since constructing the affinity matrix $W$ and computing the eigenvectors of $L$ have computational complexity $\mathcal{O}(n^2)$ and $\mathcal{O}(n^3)$ respectively. 

The Nystr\"{o}m method \citep{Williams2001} is a general method for generating good quality low rank approximations of large matrices. The Nystr\"{o}m approximation method for Spectral Clustering \citep{Fowlkes2004} randomly samples the columns of the affinity matrix $W$ and approximates the eigen decomposition of the full matrix directly using correlations between the sampled columns and the remaining columns. Effectively this can be thought of as a dial which the user has control over, sampling more columns will provide better results but at a higher computational cost. The downsides with this method are that there are no accuracy guarantees, the  memory requirements can be high, and the random sampling of columns may lead to small clusters being under represented or completely missed in the final clustering. 
%Doesn't use the affinity to choose which columns in the matrix to sample.
%Fixed complexity dependant on the number of verticies $n$ and the number of samples columns $q$
%DRINEAS/MAHONEY non-uniform sampling of the Gram matrix and bounds approximation error
%But may need to sample many columns (O(n)?)
%practicaltiy on massive datasets may be limited according to Yan. (how big is massive?)
%DRINEAS/MAHONEY non-uniform sampling of the Gram matrix and bounds approximation error
%But may need to sample many columns (O(n)?)

% Spectral Clustering is performed on the representative set only, which is significantly faster than performing Spectral Clustering on the full data set. The resulting cluster labels for the representative data are linked back to the original data set such that every original data point acquires the same label as its associated $k$-means cluster centre. 


An alternative to the Nystr\"{o}m method is to use a pre-processing technique to reduce the size of the data. A natural way to do this is to select certain representative points to summarise the whole data set.  \cite{Yan2009} proposed KASP and RASP, algorithms which use k-means and random forest methods respectively to select $q$ representative points to apply Spectral Clustering on.  Similarly \cite{Shinnou2008} also use $k$-means to identify representative points, but in addition to using these points, \cite{Shinnou2008} also include any data points which are deemed to be suitably far from any representative point in the Spectral Clustering.  In both \cite{Yan2009} and \cite{Shinnou2008} the cluster labels given to the original data points are the same as the label assigned to their nearest representative point. As an alternative, \cite{Chen2011} represents the data as a linear combination of representative points. A sequential reduction algorithm is adopted in  \cite{Chen2006a,Liu2007} with an early stopping criteria based on the observation that well separated data points converge to the final embedding more quickly. However this is only suitable for binary clustering.  Other possibilities include random projection with sampling methods \citep{Sakai2009} and shortest path methods \citep{Liu2013b}.

%\cite{Chen2011} has a ``landmark `` based SC.  $q$ landmark points are chosen to represent the data (randomly or with k-means), and the rest of the data is represented as a linear combination in a codebook type formar. % $\mathcal{O}(nq)$ and $\mathcal{O}(q^3 + q^2n)$. 

We discuss the KASP algorithm in more detail as it is the most popular speed up method for Spectral Clustering and it inspired our work in online Spectral Clustering which is introduced in Section \ref{sec:microSpec}.

In KASP, k-means is applied with $q$ clusters to the data set $X$, where $q$ is chosen such that $k \ll q \ll n$. Therefore each point in $X$ belongs to a cluster $y_j, (j \in 1, \hdots, q)$. Let the centres of these $q$ clusters be  $\widehat{y_1}, \hdots, \widehat{y_q}$. These  are used as representative points for the whole data set. Spectral Clustering is performed on the representative points, reducing the complexity of the eigen decomposition from $\mathcal{O}(n^3)$ to $\mathcal{O}(q^3)$. Finally, the original data points are assigned the cluster label  that their closest representative point $\widehat{y_j}$ was assigned in the Spectral Clustering. The KASP algorithm is given in Algorithm \ref{alg:kasp}.

\begin{algorithm}[h!]
\caption{KASP}
  \begin{algorithmic}[1]
   \REQUIRE Data set $X = {x_1,\ldots, x_n}$, number of clusters $k$, number of representative points $q$
   \ENSURE  $k$-way partition of the input data
   \STATE Perform k-means with $q$ clusters on $x_1, \hdots, x_n$ to create clusters $y_1, \hdots y_q$.
   \STATE Compute the cluster centroids $\widehat{y_1}, \hdots, \widehat{y_q}$ as the $q$ representative points.
   \STATE Build a correspondence table to associate each $x_i$ with the nearest cluster centroids $\widehat{y_j}$.
   \STATE Run a Spectral Clustering algorithm on $\widehat{y_1}, \hdots, \widehat{y_q}$ to obtain an $k$-way cluster membership
   for each of $\widehat{y_j}, (j \in 1 \hdots q)$.
   \STATE Recover the cluster membership for each $x_i$ by looking up the cluster membership of the corresponding centroid $\widehat{y_j}$ in the correspondence table.
  \end{algorithmic}
\label{alg:kasp}
\end{algorithm}

Both KASP and RASP have been shown to perform well empirically on large data sets \citep{Yan2009}, retaining good clustering performance even as the \textit{data reduction ratio} increases. We can express the data reduction ratio as $\gamma = \frac{n}{q}$. As in many of the sampling methods discussed above, in KASP the user has control over the data reduction rate. A larger value of $q$ will give a better performance but at a computational cost. The KASP authors present an upper bound on the misclustering rate given the perturbation to the original data. However as the bound depends on eigen-gaps and the misclustering rate is bound between 0 and 1, this upper bound can be weak. Finally, the method of assigning data points to clusters based on the cluster label of their representative point can lead to poor segmentation as shown in \cite{Cao2014}. They propose a local interpolation in their algorithm Local Information-based Fast Approximate Spectral Clustering (Li-ASP) to prevent this poor segmentation issue. They achieve this by assigning data points based on a weighted version of their $p$ closest representative points labels, rather than labelling based just on the label of the single closest representative point. 
 
% One constant challenge in clustering is selecting the number of clusters to search for. \cite{Zelnik-Manor2004} present a method for automatically choosing the true number of clusters using the eigenvectors to inform their choice. More commonly the eigenvalues are used to estimate the number of clusters, but if the clusters are not clearly separated identifying the number of clusters from eigenvalues alone is not trivial. We shall assume that the true number of clusters is known. 

The methods discussed above only address dealing with large data sets which are static. Our aim is to investigate methods which can update the Spectral Clustering partitioning when new data points arrive. 

\subsection{Incremental methods for Spectral Clustering}
 \label{sec:incremental}

%A data stream is a potentially endless sequence of observations obtained at high frequency relative to the available processing and storage capabilities. Data streams arise in many applications such as online purchases, modelling epidemics and understanding sensor networks. %A data stream is a potentially endless sequence of observations obtained at high frequency relative to the available processing and storage capabilities. Data streams arise in many applications such as online purchases, modelling epidemics and understanding sensor networks. 

Incremental Spectral Clustering is the problem where we have a weighted network on which to perform Spectral Clustering. New data points are added to the network over time and we wish to update the partition given by Spectral Clustering without performing a full re-clustering of the data as this may be costly and potentially not feasible.  So far there have been two different approaches to this problem (i) updating the cluster membership directly (ii) incrementally updating the eigenvectors.

The first method is described in \cite{Valgren2008}. When new points arrive, the Spectral Clustering is updated directly using a similarity threshold to assign points to clusters. If a new data point is sufficiently far from its closest representative points, it is considered the start of a new cluster. This means that the number of overall clusters  must always increase. Therefore it is not feasible for data streams. In addition there is no method for splitting existing clusters as new data points arrive. %i There are a number of problems with this method. The number of clusters (and therefore the size of the affinity matrix) must always increase. Can't deal with cluster splits. %Affinity is shrunk whenever new cluster is added.

An algorithm that incrementally updates the eigenvectors is proposed in \cite{Ning2007} and  \cite{Ning2010}. Their algorithm can deal with both additional data points joining the network and similarity weights changing between existing data points. The algorithm updates the eigenvectors and eigenvalues directly without performing a full eigen-decomposition. The addition of a new data point is treated as a series of $n$ weight changes, where is $n$ is the number of currently observed data points.  However the authors recommend a full re-clustering in batch to minimise cumulative errors. There are some issues with their update method, mainly that the updating of eigenvectors means that the orthogonality property may be lost - potentially leading to poor cluster detection. Also if the spatial neighbourhoods of often changing vertexes are large it can still be computationally difficult as the eigenvector update step involves the inversion of a matrix. Finally the authors recommend a full spectral re-clustering occasionally to prevent the accumulation of errors in the eigenvectors, this is not feasible in the streaming setting. Generally this method is not suitable for data streaming, as the size of the Laplacian can grown unbounded for an infinite data stream. %We did intend to use this algorithm as a competing algorithm in our experiments section. However, the computational costs for Ning were so great for data streaming examples that it was not possible to run the study, even when not performing a full reclustering.

Another incremental update algorithm is detailed in \cite{Dhanjal2014} which approximates the eigen decomposition of the Laplacian incrementally but still requires regular full re-clustering. %\cite{Dhanjal2011}

\cite{Kong2011} is a mixture of both Ning and Valgren's methods, using representative points like Valgren but the eigen-updating of Ning. Although it can be quicker that Ning it retains the other issues of Ning's method discussed above. In addition it has the same problem of Valgren's method that the number of clusters increases over time. This makes it unsuitable for data streams. %nh in It  It assumes sparse data set. Number of representative points will grow unbounded making it unsuitable for data streams. 

%Other variants include using fuzzy C k-means \citep{Bouchachia2012} and a model based kernel Spectral Clustering \citep{Langone2014}.

Although the methods discussed deal with some aspects of difficulties in data streams, none of them are suitable for the full problem of clustering a data stream. We introduce an online Spectral Clustering algorithm for data streams based on the Clustream model of \cite{Aggarwal2003} in Section \ref{sec:microSpec}.


\section{Online Spectral Clustering}
\label{sec:microSpec}

In this section we discuss general online clustering methods for data streams, introduce the streaming algorithm Clustream and address how to combine it with Spectral Clustering to create an online Spectral Clustering algorithm for data streams. 

\subsection{The challenges of clustering data streams}
A relatively new challenge in clustering is working with data streams \citep{Gama2010, Silva2013}. A data stream is data which arrives in an ordered sequence, continuously; for example, sensor data or online shopping transactions. There is no control over the order in which data objects should be processed. A data stream may be potentially unbounded in length, and the data points are often discarded after processing.  Much work has been done developing offline clustering methods, such as Spectral Clustering, but it is not suitable to apply these offline methods directly to the streaming scenario. Simply running an offline clustering algorithm on all the data observed so far may not be feasible for three main reasons, storage capacity, computational costs and ability to access the data.

The first challenge is storing all of the data. As a data stream is a potentially endless sequence of observations obtained at a high frequency it may not be possible to store all of the data in its entirety. Therefore the older data has to be thrown away to make room for the new arrivals which is a problem if we wish to incorporate historical data into the clustering.

Secondly, clustering algorithms can be computationally expensive. For example, computing the eigenvectors for Spectral Clustering has complexity $\mathcal{O}(n^3)$. As data streams are potentially unbounded in length, standard clustering algorithms cannot be used.  Therefore we need to be able to update our idea of the data as new points arrive efficiently and simply with little computational issues.

Finally, data streaming is often classed as a ``one-pass-access'' problem. Imagine a constant stream of data flying past your window, you can view the data as it flies by the window, but once it has passed by, it cannot be accessed again. Some traditional clustering algorithms such as DBScan \citep{Ester1996} require many passes or iterations of the data, therefore these type of clustering methods are not directly suitable for the online data streaming case.
 
\subsection{Clustream}

Clustream \citep{Aggarwal2003} offers a framework which allows quick and easy updates and the ability to perform sophisticated clustering algorithms. Clustream has proved popular, since the paper was first published in 2003 it has been cited over 1400 times. The main idea is to separate the clustering process in two stages, a micro-clustering stage and a macro-clustering stage. The  micro-clustering stage continuously updates statistical summaries of the data stream, and the macro-clustering is more computationally intensive and run in batch or on a user request. In the next subsection we detail how the micro-clustering step in the Clustream algorithm works as described in \cite{Aggarwal2003}.

\subsubsection{Micro-clustering}

The micro-clustering stage is a way of maintaining an active, evolving representative summary of the data, without storing the absolute values of the data points. Micro-clusters are defined as a temporal extension of the cluster feature vector first described in \cite{Zhang1996a}.% The data stream is summarised by many small clusters, which are initially generated by k-means. The online phase stores $q$ micro-clusters in memory, where $q$ is an input parameter.
We take an initial training set and perform k-means with $q$ clusters but choose the value of $q$ to be much larger than the expected number of true macro-clusters $k$. The aim is to create a fine scale summary of the data. The value of $q$ should be chosen to be as large as computationally comfortable. The larger $q$ is, the finer scale that the summaries will be. It is vital to ensure that the micro-clusters well represent the underlying data set or else the macro-clustering will under perform. These $q$ clusters are our first micro-clusters. Over time, we will update these micro-clusters, adding new data points to them, merging them and removing old micro-clusters, although the number of micro-clusters should stay fixed throughout. 

The micro-clusters can then be used on a user request to perform a macro-clustering using the summarised data rather than the full data set. If the micro-clusters represent the true underlying data stream well, then the difference between the clustering on the summarised data and the true full data should be small. 

Assume that we have a data stream $S$ which consists of $d$-dimensional data $\boldsymbol{x_i}$ arriving in sequence. $S = \{\boldsymbol{ x_1}, \boldsymbol{x_2}, \boldsymbol{x_3}, \hdots \boldsymbol{x_i}, \hdots, \}, \boldsymbol{x_i} \in \mathbb{R}^d$. Each micro-cluster $M_j, (j \in 1 \ldots, q)$ is stored as a ($2 \cdot d + 3$) tuple $(\boldsymbol{CF1^x_j}, \boldsymbol{CF2^x_j}, n_j, CF1^t_j, CF2^t_j)$. The definitions are given in equation \eqref{eq:microcluster_def}. $\boldsymbol{CF1^x_j}$ is the sum of all observed data in micro-cluster $j$, $\boldsymbol{CF2^x_j}$ is the sum of the squares of the data and $n_j$ is the number of elements assigned to that micro-cluster. $CF1^t_j$ and $CF2^t_j$ refer to the sum of the time stamps, and the sum of squared time stamps respectively. Note that both $\boldsymbol{CF1^x_j}$ and $\boldsymbol{CF2^x_j}$ are $d$-dimensional vectors.

Each micro-cluster $M_j$ will have 
\begin{align}
\boldsymbol{CF1^x_j} &= \quad \sum_{x_i \in M_j}{\boldsymbol{x_i}} \; , \nonumber  \\ 
\boldsymbol{CF2^x_j} &= \quad \sum_{x_i \in M_j}{(\boldsymbol{x_i})^2} \; , \nonumber\\
CF1^t_j &= \quad \sum_{i | x_i \in M_j}{t_i} \; , \nonumber   \\
CF2^t_j &= \quad\sum_{i | x_i \in M_j}{(t_i)^2} \; , \nonumber\\
n_j &= \quad \sum_{x_i \in M_j}{1} \; .
\label{eq:microcluster_def}
\end{align}

If a new data point $x_{\text{new}}$ arrives at time $t_{\text{new}}$ and is assigned to micro-cluster $M_j$, the  update  given in equation \eqref{eq:microcluster_update} is applied. 
\begin{align}
\boldsymbol{CF1^x_j} \quad &\leftarrow \quad \boldsymbol{CF1^x_j} + \boldsymbol{x}_{\text{new}} \; , \nonumber  \\ 
\boldsymbol{CF2^x_j} \quad &\leftarrow \quad \boldsymbol{CF2^x_j} + (\boldsymbol{x}_{\text{new}})^2 \; , \nonumber\\
CF1^t_j \quad &\leftarrow \quad  CF1^t_j + t_{\text{new}} \; , \nonumber   \\
CF2^t_j \quad &\leftarrow \quad CF2^t_j + (t_{\text{new}})^2 \; , \nonumber\\
n_j  \quad &\leftarrow \quad n_j + 1 \; .
\label{eq:microcluster_update}
\end{align}

Note that updating the micro-clusters requires only addition therefore updating is computationally efficient. Critically it is possible to use these summaries to calculate the centre of each micro-cluster as in equation \eqref{eq:micro_centre}. 

\begin{equation}
  \label{eq:micro_centre}
  \text{Centre of micro-cluster j}  = \boldsymbol{\bar{M_j}} = \frac{\boldsymbol{CF1^x_j}}{n_j}
\end{equation}

It is these centres which as used as representative points for input into the macro-clustering.  As new points in the data stream arrive, they are either allocated to a micro-cluster and the update procedure discussed above is carried out, or a new micro-cluster is created. The decision for a new micro-cluster to be created is based on whether the new data point is close enough to it's nearest cluster centre. 

When a new data point arrives it's nearest micro-cluster $M_{*}$ is identified using the Euclidean distance metric given in equation \eqref{eq:m*}.

\begin{equation}
  \label{eq:m*}
 M_{*} = \argmin_{M_j , j \in 1:q} \lVert {\boldsymbol{x_i} -\boldsymbol{\bar{M_j}} } \rVert ^2
\end{equation}

 If the data point falls within the maximum boundary factor of it's nearest cluster centre, then it is absorbed as part of that cluster. If not, it is used to create a new micro-cluster. In Clustream, the maximum boundary factor is defined as a factor of $t$ of the root-mean-square deviation of the data points in $M_j$ from the centroid of $M_j$. However we stated earlier than the number of micro-clusters must remain fixed throughout the process. Therefore if a new micro-cluster is formed, either an existing micro-cluster must be deleted, or two close micro-clusters should be merged.  %The root-mean-square deviation can only be defined for a cluster with more than one point. For a cluster with only one previous point, a heuristic is used to define the MBF. Details of the MBF can be found in \cite{Aggarwal2003}%  Paper uses r times that of the next closest cluster. Should I write MBF mathematically? Maybe show it visually with the cuboids.

 We follow the methodology in Clustream by first looking for an old micro-cluster to delete and otherwise combine the two nearest micro-clusters. The first step is to see if an existing micro-cluster can be deleted to make room for the new micro-cluster. The criteria for deleting micro-clusters is their \textit{relevancy}. Clustream approximates the average timestamp of the last $m$ data points of the cluster $M_j$ (where $m$ is a user chosen parameter) and judges if the cluster is old enough to discard.  Let the mean and standard deviation of the arrival times for a micro-cluster $M_j$ be given by $\mu M_j$ and $\sigma M_j$. These can easily be calculated as we store $CF1^t$ and $CF2^t$. The \textit{relevancy stamp} $r(M_j)$ is defined to be the arrival of the $(m/2n_j)^{th}$ percentile of the points in $M_j$ assuming the timestamps are Normally distributed. We check if the micro-cluster with the smallest relevancy stamp has $r(M_j) < \delta$, where $\delta$ is some user-chosen deletion threshold as given in equation \eqref{eq:clustream_deletion}.

\begin{equation}
  \label{eq:clustream_deletion}
 \argmin_{M_j , j \in 1:q}(r(M_j)) < \delta  
\end{equation}

 If the inequality in equation \eqref{eq:clustream_deletion} holds then the  micro-cluster with the minimum relevancy stamp is deleted. If not, then no micro-clusters are deleted and instead the two closest micro-clusters are merged. If two micro-clusters $M_r$ and $M_s$ are to be merged, the updates given in equation \eqref{eq:microcluster_merge} are used to merge them into $M_r$, and $M_s$ will be deleted. Again as all of these updates only involve addition steps, they are fast to implement. 
\begin{align}
\boldsymbol{CF1^x_r} \quad &\leftarrow \quad \boldsymbol{CF1^x_r} + \boldsymbol{CF1^x_s} \; , \nonumber  \\ 
\boldsymbol{CF2^x_r} \quad &\leftarrow \quad \boldsymbol{CF2^x_r} + \boldsymbol{CF2^x_s} \; , \nonumber\\
CF1^t_r \quad &\leftarrow \quad  CF1^t_r + CF1^t_s\; , \nonumber   \\
CF2^t_r \quad &\leftarrow \quad CF2^t_r +  CF2^t_s\; , \nonumber\\
n_r  \quad &\leftarrow \quad n_r + n_s \; .
\label{eq:microcluster_merge}
\end{align}

With this online micro-cluster maintenance, the data stream should remain well represented over time. The micro-clustering update algorithm for Clustream is given in Algorithm \ref{alg:clustream}.

 %When a new data point arrives, if it is the start of a new evolving cluster it will be allowed to grow however if it is an outlier no more points will be added to it and over time it may be deleted from the system all together.

\begin{algorithm}
\caption{Clustream Micro-clustering}  
\begin{algorithmic}
\REQUIRE Data Stream $S = {\boldsymbol{x_1},\ldots, \boldsymbol{x_n}  }$, number of micro-clusters $q$
\ENSURE Micro-clusters $M_1, \ldots, M_q$
\STATE Initialise the micro-clusters k-means($x_1, \hdots x_{init},q$) and equations \eqref{eq:microcluster_def}
\FOR {each new data point $x_i$}
 \STATE Find the closest micro-cluster to $x_i$, $M_*$ using equation \eqref{eq:m*}
 \IF{$x_i$ falls within the maximum boundary for $M_*$ }
   \STATE absorb $x_i$ into micro-cluster $M_*$ using equations \eqref{eq:microcluster_update}
 \ELSE
 \STATE Use $x_i$ to initialise it's own new micro-cluster using equations \eqref{eq:microcluster_def}
  \IF{any micro-cluster is suitably old according to equation \eqref{eq:clustream_deletion}}
   \STATE Remove the oldest micro-cluster
  \ELSE 
   \STATE Merge the two closest micro-clusters using equation \eqref{eq:microcluster_merge}
  \ENDIF
\ENDIF
\ENDFOR
\end{algorithmic}
\label{alg:clustream}
\end{algorithm}
%Thus, the CluStream algorithm finds the arrival time (known as the relevance time) of the m/(2Ni)th percentile of the Ni objects in a micro-cluster i, whose timestamps are assumed to be normally distributed.
\subsubsection{Macro-clustering stage}
 
The second stage of clustream is a macro-clustering stage, where we take the current micro-cluster feature vectors, and use these as input into global clustering algorithm. The macro-clustering step is where the general data summary is transformed into a snapshot of the true underlying clusters at that point in the stream. The $q$ micro-cluster centres $\bar{M_j},(1 \leq j \leq q$) are treated as representative points for the data stream $S$, and a standard clustering algorithm can be used to determine clusters. By using the micro-clusters to summarise the data, we can therefore perform Spectral Clustering on data streams.  The full algorithm is given in Algorithm \ref{alg:onlineSpec}. % The nature of this algorithm allows the user to get close to online streaming and perform Spectral Clustering on a summary of the whole of the data set. 

%In order to achieve this we adopt a micro-clustering type approach to quickly update a summary of the data. When an overall clustering is required, Spectral Clustering is performed using the centres of the micro-clusters as the input data. The micro-clusters act as a way of summarising the constantly arriving data steam whilst allowing updates to occur in a non intrusive, non-computationally difficult manner, with limited storage requirements. 

\begin{algorithm}
\caption{Spectral Clustream}
\begin{algorithmic}[1]
\REQUIRE Data Stream $S = {\boldsymbol{x_1},\ldots, \boldsymbol{x_n}  }$, number of clusters $k$, number of micro-clusters $q$
\ENSURE A $k$ way clustering of the micro-clusters $M_1, \ldots, M_q$.
\STATE Initialise the micro-clusters using k-means($x_1, \hdots x_{init},q$) and equations \eqref{eq:microcluster_def}
\FOR {each new data point $x_i$}
 \STATE Apply Clustream update as in Algorithm \ref{alg:clustream}
 \IF{A Macro-clustering is required}
   \STATE Perform Spectral Clustering on $M_1, \ldots, M_q$ with $k$ clusters.
\ENDIF
\ENDFOR

\end{algorithmic}
\label{alg:onlineSpec}
\end{algorithm}

There are a couple of possible ways to feed the micro-clusters into a Spectral Clustering algorithm (step 5 of Algorithm \ref{alg:onlineSpec}). Two of the options suggested in \cite{Zhang1996a} listed here. 

\begin{enumerate}
\item Calculate the centre of each micro-cluster $\bar{M_j}$ and use it as an object to be clustered by the macro-clustering algorithm.
\item Do the same as before, but weighting each micro-cluster centre $\bar{M_j}$ proportionally to $n_j$, the number of points assigned to that micro-cluster, so that micro-clusters with more objects will have more influence on the final clustering.
\end{enumerate}

No guidance is given in \cite{Zhang1996a} to how these two different approaches might affect the final clustering result. Next, in Section \ref{sec:weighting} we describe how to weight the micro-cluster centers. Later in Section \ref{sec:clustream_exp} we will analyse the performance of both unweighted and weighted Online Spectral Clustering.

\subsection{Weighting the Micro-Clusters}
\label{sec:weighting}

In this section, we discuss how can create a weighted affinity matrix, look at the effect this has on the Laplacians and note a spectral link between weighting in this manner and using larger affinity consisting of repeated points.
Weighting the micro-clusters is suggested in \cite{Zhang1996a} but why might it be beneficial to weight the micro-clusters?  The first thing to note is that the number of data points assigned to micro-clusters is not uniform across the micro-clusters, and this distribution will change as the stream progresses.  For example Figure \ref{fig:hist1} shows a histogram of the number of points assigned to micro-clusters at the start of a data stream, Figure \ref{fig:hist2} shows the middle of the stream, and Figure \ref{fig:hist3} shows the end of the stream. We can see that the distribution is not uniform. Therefore some information is contained in the number of points assigned to a micro-cluster. 

\begin{figure}[h!]
  \centering
  \begin{subfigure}{0.32\textwidth}
    \centering
    \includegraphics[width = \textwidth]{microcluster_histograms/s_set_1_hist_microclusters_time_501_fixed.png}
    \caption{Start of the stream}
  \label{fig:hist1}
  \end{subfigure}
  \begin{subfigure}{0.32\textwidth}
    \centering
    \includegraphics[width = \textwidth]{microcluster_histograms/s_set_1_hist_microclusters_time_1250_fixed.png}
  \caption{Middle of the stream}
  \label{fig:hist2}
  \end{subfigure}
\begin{subfigure}{0.32\textwidth}
    \centering
    \includegraphics[width = \textwidth]{microcluster_histograms/s_set_1_hist_microclusters_time_2000_fixed.png}
  \caption{End of the stream}
  \label{fig:hist3}
  \end{subfigure}
    \caption{Histograms showing the number of points assigned to micro-clusters}
  \label{fig:microHist}
\end{figure}

Secondly, imagine the scenario pictured in Figure \ref{fig:motivate_weighting} where we have two clusters, one much more dense that then other.  In the example, many micro-clusters are used to represent the  cluster on the right, although each micro-cluster only has a few data points assigned to it. The more dense cluster in the bottom left of the plot has only 3 micro-clusters representing it, but each micro-cluster has hundred of data points assigned to it.   Weighting by the number of points assigned to a micro-cluster may help balance out this scenario for the Spectral Clustering stage. 

\begin{figure}[h]
  \centering
  \includegraphics[width = 7cm]{motivate_weighting.png}
  \caption{Possible micro-cluster  locations in a toy example.}
\label{fig:motivate_weighting}
\end{figure}

In order to weight the the micro-clusters, we simply construct an affinity as described. 
Let $W \in \mathbb{R}^{q \times q}$ be the affinity matrix of the micro-cluster centres with $i,j$-th element equal to the similarity between micro-cluster $M_i$ and $M_j$, 

\[ W_{i,j} = \exp \left(- \frac{\| \bar{M_i} - \bar{M_j}\|^2}{2 \sigma^2} \right), \quad i, j = 1, \ldots, q. \]

Define the \textit{weighted affinity matrix} to be $\tilde{W} \in \mathbb{R}^{q \times q}$ where $ \tilde{W_{ij}} = n_in_jW_{ij}$. We can see that $\tilde{W}$ is a valid affinity matrix since it is symmetric with non-negative entries. If we wish to have $\tilde{W}_{ij} \leq 1$ then simply divide $\tilde{W}$ by $\argmax_i n_i^2$, but this makes no difference to the spectral decomposition \citep{Luxburg2008}. 

There exists a link between the spectral decomposition of  the Laplacian generated by $\tilde{W}$ and the Laplacian arising from a data set of repeated points, which we define as follows.  Let $W^* \in \mathbb{R}^{n \times n}$ be the repeated affinity matrix with the micro-cluster centres repeated based on the number of points assigned to them. Assume that the columns (and therefore rows) of $W^*$ are ordered such that the first $n_1$ are associated with the data assigned to micro-cluster 1, which has size $n_1$ and the next $n_2$ with those assigned to micro-cluster 2, and so on. Let $D, \tilde{D}, D^{*},$ be the corresponding degree matrices and $L, \tilde{L}, L^{*}$ be the corresponding normalised symmetric Laplacians.

Lets consider the Affinity and Laplacian matrices more closely for a very simple case. Assume that we have two micro-clusters, $M_1$ and $M_2$,  which have $n_1$ and $n_2$ points assigned to them respectively. Let the similarity between the two micro-cluster centres be $s$, and assume that we are using the standard Gaussian kernel to generate affinity matrices, so therefore the diagonal elements will be equal to 1. The affinity, degree and Laplacian matrices (W,D and L) for the two micro-cluster centres are given in equation \eqref{eq:example_centers}.

\begin{equation}
  \label{eq:example_centers}
 W = \left(
  \begin{array}{cc}
    1 & s \\
    s & 1
  \end{array} \right), \quad
%
 D = \left(
  \begin{array}{cc}
    1+s & 0 \\
    0 & 1+s
  \end{array} \right), \quad
%
 L_{\text{symm}} = \left(
  \begin{array}{cc}
    \frac{1}{1+s} & \frac{s}{1+s} \\
    \frac{s}{1+s} & \frac{1}{1+s} 
  \end{array} \right) 
\end{equation}

In order to create a weighted version of the affinity matrix, we simply multiply through by $n_1$ and $n_2$. The weighted affinity matrix $\tilde{W}$ and related degree and Laplacians ($\tilde{D}$ and $\tilde{L}$) are given in equation \eqref{eq:example_weighted}.% We can see how this is incorporated into the Laplacian. Note that we are working with the Symmetric Normalised Laplacian. In order to weight in this manner for the unnormalised Laplacian a scaling of the diagonal matrix is would be required  \citep{Luxburg2008}.

\begin{equation}
\begin{gathered}
 \tilde{W} = \left(
  \begin{array}{cc}
    n_1^2 & sn_1n_2 \\
    sn_1n_2 & n_2^2
  \end{array} \right) , \quad 
%
 \tilde{D} = \left(
  \begin{array}{cc}
    n_1^2 + sn_1n_2 & 0 \\
    0 & n_2^2 + sn_1n_2
  \end{array} \right) \\ \\ 
 \tilde{L}_{\text{symm}} = \left(
  \begin{array}{cc}
   \frac{n_1^2}{n_1^2+sn_1n_2} & \frac{sn_1n_2}{\sqrt{(n_1^2+sn_1n_2)(n_2^2+sn_1n_2)}} \\
   \frac{sn_1n_2}{\sqrt{(n_1^2+sn_1n_2)(n_2^2+sn_1n_2)}} &  \frac{n_2^2}{n_2^2+sn_1n_2}
  \end{array} \right) \\
\end{gathered} \label{eq:example_weighted} 
\end{equation}

Finally we observe the construction of the repeated affinity matrix given in equation \eqref{eq:example_repeated}. Here the block nature in $W^*$, $D^*$ and $L^*$ is clear. The first $n_1$ rows of $W^*$ relate to the centre of micro-cluster 1, and the bottom $n_2$ rows relate to the centre of micro-cluster 2.

\begin{equation}
\begin{gathered}
  W^* = \left(
  \begin{array}{cccccc}
    1 & \hdots & 1 & s & \hdots & s \\
    \vdots & \ddots & \vdots & \vdots & \ddots & \vdots \\
    1 & \hdots & 1 & s & \hdots & s \\
   s & \hdots & s & 1 & \hdots & 1 \\
    \vdots & \ddots & \vdots & \vdots & \ddots & \vdots \\
    s & \hdots & s & 1 & \hdots & 1 
  \end{array} \right) , \quad 
%
 D^* = \left(
  \begin{array}{cccccc}
    \star & 0 & 0 & 0 & \hdots & 0 \\
    0 & \ddots & 0 & \vdots & \ddots & \vdots \\
    0 & 0 & \star & 0 & \hdots & 0 \\
   0 & \hdots & 0 & \bigtriangleup & 0 & 0 \\
    \vdots & \ddots & \vdots & 0 & \ddots & 0 \\
    0 & \hdots & 0 & 0 & 0 & \bigtriangleup 
  \end{array} \right) \\
%
 L_{\text{symm}}^* = \left(
  \begin{array}{cccccc}
    \frac{1}{\star} & \hdots & \frac{1}{\star} & \frac{s}{\sqrt{\star \bigtriangleup}} & \hdots & \frac{s}{\sqrt{\star \bigtriangleup}} \\
    \vdots & \ddots & \vdots & \vdots & \ddots & \vdots \\
    \frac{1}{\star} & \hdots & \frac{1}{\star} & \frac{s}{\sqrt{\star \bigtriangleup}} & \hdots & \frac{s}{\sqrt{\star \bigtriangleup}} \\
   \frac{s}{\sqrt{\star \bigtriangleup}} & \hdots & \frac{s}{\sqrt{\star \bigtriangleup}} & \frac{1}{\bigtriangleup}  & \hdots &  \frac{1}{\bigtriangleup} \\
    \vdots & \ddots & \vdots & \vdots & \ddots & \vdots \\
    \frac{s}{\sqrt{\star \bigtriangleup}} & \hdots & \frac{s}{\sqrt{\star \bigtriangleup}} & \frac{1}{\bigtriangleup}  & \hdots &  \frac{1}{\bigtriangleup} 
  \end{array} \right) 
\\
\text{where } \star = n_1 + n_2s \text{ and } \bigtriangleup = n_1s + n_2. 
\end{gathered} \label{eq:example_repeated}
\end{equation}

If we evaluate these expressions for a particular numerical case, we can see how the spectral decomposition of the matrices $\tilde{L}$ and $L^* $are linked.

Let $s = 0.5$, $n_1 = 3$, $n_2 = 2$. The 2nd smallest eigenvector of $L^*_{\text{symm}}$ is 

\begin{equation}
  \label{eq:e2*}
   e_2^* = \left[ 
\begin{array}{ccccc}
  -0.350& -0.350& -0.350& 0.562& 0.562  
\end{array} \right].
\end{equation}

The 2nd smallest eigenvector of $\tilde{L}_{\text{symm}}$ is 

\begin{equation}
  \label{eq:e2}
   \tilde{e}_2 = \left[ 
\begin{array}{cc}
  0.607 & -0.795 \\
\end{array} \right] .
\end{equation}

If we expand the eigenvector $\tilde{e}_2$ by expanding it's elements and block dividing by $\sqrt{n_1}$ and $\sqrt{n_2}$ respectively, we get the following, 
\begin{equation}
  \label{eq:e_repeat}
   \left[ \; \overbrace{\frac{0.607}{\sqrt{3}} \; \frac{0.607}{\sqrt{3}} \; \frac{0.607}{\sqrt{3}}}^{n_1} \; \overbrace{\frac{-0.795}{\sqrt{2}} \; \frac{-0.795}{\sqrt{2}}}^{n_2} \; \right] 
=  \left[ 
\begin{array}{ccccc}
0.350& 0.350& 0.350 & -0.562& -0.562  
\end{array} \right] .
\end{equation}

We can see that the right hand vector in equation \eqref{eq:e_repeat} is the negative of the 2nd smallest eigenvector of $L^*_{\text{symm}}$, $e_2^* $ (equation \eqref{eq:e2*}). 
In fact, we will always have that the expanded repeated eigenvector of the weighted Laplacian equal to either $e_k^*$ or $-e_k^*$ for all $k$.

\RD{Nicos - Please can you help me explain the following proof better? } 
Generally, the spectral decomposition of $L^*$ and $\tilde{L}$ are linked in this way, and therefore the partition generated by performing Spectral Clustering on the weighted micro-cluster centers will be the same as the partition generated by performing Spectral Clustering on a set of repeated micro-cluster centers. We do not show a full proof for this here but sketch out the proof showing that the second smallest eigenvectors are linked in this way. 

 Let $\tilde{u}$ be the second smallest eigenvector of $\tilde{L}$.  Define $u^* =  \left( \overbrace{\frac{\tilde{u}_1}{\sqrt{n_1}}, \; \frac{\tilde{u}_1}{\sqrt{n_1}}, \; \frac{\tilde{u}_1}{\sqrt{n_1}}}^{n_1}, \; \hdots, \overbrace{\frac{\tilde{u}_k}{\sqrt{n_k}}, \; \frac{\tilde{u}_k}{\sqrt{n_k}}}^{n_k} \right)$.  We wish to show that $u^*$ is the second smallest eigenvector of $L^*$.  First we will show that $u^*$ satisfies the following criteria, 
%We can see that $\|u^*\|^2 = \sum_{i = 1}^{n}(u_i^*)^2 = \sum_{i=1}^{k}n_i \cdot \frac{\tilde{u}_i^2}{n_i} = \sum_{i = 1}^{k} \tilde{u}^2_i = 1$. 

\begin{itemize}
\setlength\itemsep{0.5em}
\item[(i)] $ \|u^*\| = 1 $
\item[(ii)] $u^* \perp D^{* 1/2} \mathds{1}$
\item[(iii)] $u^{*\top} L^* u^* = \tilde{u}^{\top} \tilde{L} \tilde{u}$
\end{itemize}

%Part (i) checks that $u^*$ has size 1 which is required for eigenvectors. Part (ii) checks that $u$ is orthogonal to the smallest eigenvector. Part (iii) shows that. 
Part(i)\\
\begin{align*}
 \|u^*\|^2 &= \sum_{i=1}^{n}u^{* 2}_i \\
&= \sum_{i = 1}^{k}n_i \frac{\tilde{u}_i^2}{n_i} \\
&= \sum_{i=1}^{k}\tilde{u}_i^2 = 1 
\end{align*}
since $\tilde{u}$ is an eigenvector. Therefore $\|u^*\| = 1$. 
 

Part (ii) \\
\begin{align*}
 \sum_{i=1}^nu^*_i D_{ii}^{* 1/2} &= \sum_{i=1}^k n_i \frac{\tilde{u}_i}{\sqrt{n_i}} \frac{\tilde{D}^{1/2}_{ii}}{\sqrt{n_i}} \\
&= \sum_{i=1}^k \tilde{u}_i \tilde{D}^{1/2}_{ii} = 0
\end{align*}
since $\tilde{u} \perp \tilde{D}^{1/2} \mathds{1}$, therefore $u^* \perp D^{* 1/2} \mathds{1}.$ 

Part (iii) \\
First we state the general property given in \cite{Luxburg2008} for the normalised Laplacian. For every $f \in  \mathbb{R}^n$ we have
\begin{equation}
  \label{eq:luxburg_vectors}
  f^{\prime} L_{\text{symm}}f = \frac{1}{2} \sum_{i,j=1}^n w_{ij} \left( \frac{f_i}{\sqrt{d_i}} - \frac{f_j}{\sqrt{d_j}}  \right)^2.
\end{equation}

Using this property, 
\begin{align*}
  \tilde{u}^{\top} \tilde{L} \tilde{u} &= \frac{1}{2} \sum_{i = 1}^k \sum_{j = 1}^k \left( \frac{\tilde{u}_i}{\sqrt{\tilde{D}_{ii \phantom{j}}}} - \frac{\tilde{u}_j}{\sqrt{\tilde{D}_{jj}}}  \right)^2 \tilde{W}_{ij} \\
 &= \frac{1}{2} \sum_{i = 1}^k \sum_{j = 1}^k \left( \frac{\tilde{u}_i / \sqrt{n_i}}{\sqrt{D_{ii}}} - \frac{\tilde{u}_j / \sqrt{n_j}}{\sqrt{D_{jj}}} \right)^2 n_i n_j  W_{ij}
  %&= \frac{1}{2} \sum_{i = 1}^k \sum_{j = 1}^k \left( \frac{\tilde{u}_i}{\sqrt{\sum_{l=1}^kn_ln_iA^{\prime}_{ij}}} - \frac{\tilde{u}_j}{\sqrt{\sum_{l=1}^kn_ln_jA^{\prime}_{ij}}} \right)^2 \tilde{A}^{\prime}_{ij}
\end{align*}
which we can see is equal to $ u^{* \top}L^*u^*$ by observing that $W^*$ has repeated elements from $W$.

Now that all of the above criteria have been satisfied, we know that $u^*$ is an eigenvector of $L^*$ and is orthogonal to the smallest eigenvector. Assume there exists some $v^*$ such that $v^* \perp D^{* 1/2}\mathds{1}$ with $\|v^*\|=1$ and $v^{* \top}L^*v^* < u^{* \top}L^*u^*$. Then $\exists \; \tilde{v}$ with $\tilde{v}^{\top} \perp \tilde{D}^{1/2}\mathds{1}$ and $\tilde{v}^T \tilde{L} \tilde{v} < \tilde{u}^{\top} \tilde{L} \tilde{u}$. This contradicts the fact that $\tilde{u}$ is the second smallest eigenvector of $\tilde{L}$. Therefore $u^*$ is the second smallest eigenvector of $L^*$ \qed 
%Now we check that there does not exist some other $v^*$ which is also a eigenvector of $L^*$ which is linked to a smaller eigenvalue of $L^*$ than $u^*$. 

\section{Experimentation}
\label{sec:clustream_exp}

In this section, we investigate the performance of both Unweighted and Weighted Spectral Clustream and a simple windowed approach to Spectral Clustering. First the algorithms and methodology are introduced and the performance metrics defined. Then the algorithms are compared on simulated data, two real image based data sets and an evolving data set. 

\subsubsection{The Algorithms}

Spectral Clustream is given in Algorithm \ref{alg:onlineSpec}. As described in Section \ref{sec:microSpec}, there are two  ways that we can incorporate the micro-cluster centres into the macro-clustering; not weighting, or weighting. Unweighted clustream takes the micro-cluster centres as direct input into the Spectral Clustering algorithm. Weighted clustream weights the micro-cluster centres by the number of data points assigned to that micro-cluster. In both Spectral Clustream algorithms we use 150 micro-clusters to summarise the data stream.

Windowed Spectral Clustering is a simple algorithm which retains a window of only the  $w$  most recently observed data points. When a new data point is observed, the oldest data point in the window is discarded to make room for the new data point. Only the data points in the current window are available for input into a standard Spectral Clustering algorithm. We use a fixed window size of $150$ data points for the duration of the stream.

\subsection{Methodology}
\label{sec:methodology}

A simulated data stream $S = {\boldsymbol{x_1},\ldots, \boldsymbol{x_n}  }$ is observed sequentially, with one data point $\boldsymbol{x_t}$ observed at each time point $t \in 1, \ldots, n$. Spectral Clustream requires the micro-clusters to be initialised.  This is achieved by applying k-means with 150 clusters to the first 500 points. The initial micro-cluster components are then calculated using the equations described in equation \eqref{eq:microcluster_def}. %Spectral Windowed uses a window of $150$ data points to initialise the data window.  

At each time point $t$, a new data point  $\boldsymbol{x_t}$ is observed and is used to update the streaming algorithms. For Spectral Clustream, this means updating the micro-clusters as outlined in Algorithm \ref{alg:clustream}. Windowed Spectral Clustering shifts the window along by one, forgetting the oldest data point $\boldsymbol{x_{(t-150)}}$ and including $\boldsymbol{x_t}$. 

Every ten time steps we use the current snapshot of the data to apply Spectral Clustering and generate cluster labels for the data. This is achieved by applying Spectral Clustering to the \textit{representative points} of the data stream. In Spectral Clustream the representative points are the micro-clusters (weight adjusted or not). In Windowed Spectral Clustering the representative points are the $w$ data points currently in the window.

We then evaluate the Spectral Clustering performance. The following step is repeated 10 times and the results are averaged out. Define the \textit{test set} to be the next 200 data points in the data set( $\boldsymbol{x_{t+1}}$ , \ldots  $\boldsymbol{x_{t+200}}$). Assign each of the 200 points to their nearest representative points using the Euclidean distance metric. The test data are assigned the same cluster label that their representative point has. Performance is measured in terms of purity and V-measure of the test data using the known true clusters of the simulated data. Purity and V-measure are defined in the next section.

%In Spectral Clustream this is done by feeding the centres of the micro-clusters (weight adjusted or not) into a standard Spectral Clustering algorithm. In windowed Spectral Clustering, all  $w$ of the data points in the window as used as input into a standard Spectral Clustering algorithm. 


%Simulate a new test set of 200 data points using the underlying, unknown data stream parameters at time $t$. 
%Note that when comparing performance on the real data sets obviously we cannot simulate from the underlying distribution.
% Instead we look forward to the next 200 data points, assign them using nearest neighbour and continue in the manner above.

\subsection{Performance Measures}
\label{sec:performance}
The two measures that are used to quantify cluster performance are purity and V-measure, both of which are well used in the clustering literature. Both measures require knowledge of the ``true class'' of the data points, which may not always be available for real data sets. 

Let N be the number of data points, $C = \left\{ c_i | 1, \ldots n  \right\} $ be the set of true classes and $K = \left\{ k_j | 1, \ldots m \right\}$ be the set of clusters assigned by the clustering algorithm.  Define $A$  to be  the contingency table produced by the clustering algorithm representing the clustering solution such that $a_{ij}$ is the number of points that are members of class $c_i$ and assigned to cluster $k_j$.

Purity is the more intuitive of the two measures to understand as it quantifies how pure clusters are. A cluster which only contains data points associated with one true class will be given a high purity value. A cluster which consists of data points from many different true classes will receive a low purity value.  Purity is calculated as follows. For each cluster find the true class which is most prevalent in that cluster and count how many data points of that class there are in that cluster. Do this for all clusters, sum these counts and divide by the total number of data points. This is shown mathematically in equation \eqref{eq:purity}.

\begin{equation}
  \label{eq:purity}
  %\text{Purity} = \frac{1}{N} \sum_{k}\max_j | \omega_k \cap c_j |,
  %\text{Purity} = \frac{1}{N} \sum_{j}\max_i | k_j \cap c_i |.
  \text{Purity} = \frac{1}{N} \sum_k \argmax_c (a_{ck}).
\end{equation}
Generally this is a useful measure, however if we had a data set where each data point was assigned to a different cluster then the purity would be perfect even though this isn't a particularly good clustering of the data. Purity can be unreliable if the number of clusters is much larger than the number of true classes. 

To account for this we also use the V-measure \citep{Rosenberg2007} which  takes the harmonic mean of two other performance measures, homogeneity and completeness. Homogeneity assesses if each cluster contains members of only a single class (in a similar way to purity), whilst completeness checks that all members of the same class are assigned to the same cluster. This is shown in equation \eqref{eq:vmeasure}:

\begin{equation}
  \label{eq:vmeasure}
  \text{V-measure} = 2 \frac{h \times c}{h + c} \;,
\end{equation}

where $h$ and $c$ are homogeneity and completeness measures as defined below in equation  \eqref{eq:homogeneity} and equation \eqref{eq:completeness} respectively. 

For homogeneity to be perfect, the clustering algorithm must assign only those data points that are members of a single class to a single cluster.
\begin{equation}
\label{eq:homogeneity}
  h=\left\{
  \begin{array}{@{}ll@{}}
    1  & \text{if} \;  H(C|K) = 0 \\
    1 - \frac{H(C|K)}{H(C)}, & \text{else}
  \end{array}\right.
\end{equation} 

\begin{align}
  \label{eq:H(C|K)}
H(C|K) &=  - \sum_{k=1}^{|K|} \sum_{c=1}^{|C|}\frac{a_{ck}}{N} \log \frac{a_{ck}}{\sum_{c=1}^{|C|}a_{ck}}\\
H(C) &=  - \sum_{c=1}^{|C|} \frac{\sum_{k=1}^{|K|}a_{ck}}{N} \log \frac{\sum_{k=1}^{|K|}a_{ck}}{N}
\end{align}

For perfect completeness the clustering must assign all data points which are members of a single class to a single cluster.

\begin{equation}
\label{eq:completeness}
 c=\left\{
  \begin{array}{@{}ll@{}}
    1  & \text{if} \;  H(K|C) = 0 \\
    1 - \frac{H(K|C)}{H(K)}, & \text{else}
  \end{array}\right.
\end{equation} 

\begin{align}
  \label{eq:H(K|C)}
H(K|C) &=  - \sum_{c=1}^{|C|} \sum_{k=1}^{|K|}\frac{a_{ck}}{N} \log \frac{a_{ck}}{\sum_{k=1}^{|K|}a_{ck}}\\
H(K) &=  - \sum_{k=1}^{|K|} \frac{\sum_{c=1}^{|C|}a_{ck}}{N} \log \frac{\sum_{c=1}^{|C|}a_{ck}}{N}
\end{align}

Both purity and V-measure are bound between 0 and 1, where 1 indicates perfect performance. 


\subsection{Simulated Results}

The first data sets tested are the popular S-sets, first introduced in \cite{Franti2006}. The sets consist of synthetic 2d data with n=5000 vectors and k=15 Gaussian clusters with different degrees of cluster overlapping.   The four data sets are shown in Figure \ref{fig:s_set_truth}. Set S1 is fairly easy to cluster as all 15 clusters  are well separated. The sets become increasingly more challenging and S4 can be difficult even for humans to separate correctly.

\begin{figure}[H]
\centering
\begin{subfigure}{.4\textwidth}
  \centering
  \includegraphics[width=.8\linewidth]{s_set/s_set_1_truth.png}
  \caption{S1}
 % \label{fig:sfig1}
\end{subfigure}%
\begin{subfigure}{.4\textwidth}
  \centering
  \includegraphics[width=.8\linewidth]{s_set/s_set_2_truth.png}
  \caption{S2}
%  \label{fig:sfig2}
\end{subfigure}

\begin{subfigure}{.4\textwidth}
  \centering
  \includegraphics[width=.8\linewidth]{s_set/s_set_3_truth.png}
  \caption{S3}
 % \label{fig:sfig1}
\end{subfigure}%
\begin{subfigure}{.4\textwidth}
  \centering
  \includegraphics[width=.8\linewidth]{s_set/s_set_4_truth.png}
  \caption{S4}
%  \label{fig:sfig2}
\end{subfigure}
\caption{The S sets with true labels shown}
\label{fig:s_set_truth}
\end{figure}


The results from these sets are shown in Figure \ref{fig:s_set_results}. The data sets S1, \ldots, S4 are shown in the rows. The first column shows purity and the second column shows V-measure. The x-axis shows the batch number and therefore we can see the performance changing over time. However these data sets are static so we would not expect to see performance change with time. The shaded area depicts the inter quartile range. We can see that the unweighted Spectral Clustream (green) and windowed Spectral (blue) perform similarly for all sets. As expected, performance of all algorithms deteriorate as the data sets become harder, with more cluster overlap.  The Weighted Spectral Clustream (red) initially starts with performance on par with the competing algorithms, but quickly drops to poor performance and does not recover. Given that the underlying distributions for the S sets are stationary, this behaviour is unusual.  

\begin{figure}[H]
\begin{subfigure}{.45\textwidth}
  \centering
  \includegraphics[width=.7\linewidth]{s_set/s_set_1_ci_one_size_purity.png}
  \caption{S1 Purity}
% \label{fig:sfig1}
\end{subfigure}%
\begin{subfigure}{.45\textwidth}
  \centering
  \includegraphics[width=.7\linewidth]{s_set/s_set_1_ci_one_size_vmeasure.png}
  \caption{S1 V measure}
%  \label{fig:sfig2}
\end{subfigure}

\begin{subfigure}{.45\textwidth}
  \centering
  \includegraphics[width=.7\linewidth]{s_set/s_set_2_ci_one_size_purity.png}
  \caption{S2 Purity}
 % \label{fig:sfig1}
\end{subfigure}%
\begin{subfigure}{.45\textwidth}
  \centering
  \includegraphics[width=.7\linewidth]{s_set/s_set_2_ci_one_size_vmeasure.png}
  \caption{S2 V measure}
%  \label{fig:sfig2}
\end{subfigure}
\begin{subfigure}{.45\textwidth}
  \centering
  \includegraphics[width=.7\linewidth]{s_set/s_set_3_ci_one_size_purity.png}
  \caption{S3 Purity}
 % \label{fig:sfig1}
\end{subfigure}%
\begin{subfigure}{.45\textwidth}
  \centering
  \includegraphics[width=.7\linewidth]{s_set/s_set_3_ci_one_size_vmeasure.png}
  \caption{S3 V measure}
%  \label{fig:sfig2}
\end{subfigure}
\begin{subfigure}{.45\textwidth}
  \centering
  \includegraphics[width=.7\linewidth]{s_set/s_set_4_ci_one_size_purity.png}
  \caption{S4 Purity}
 % \label{fig:sfig1}
\end{subfigure}%
\begin{subfigure}{.45\textwidth}
  \centering
  \includegraphics[width=.7\linewidth]{s_set/s_set_4_ci_one_size_vmeasure.png}
  \caption{S4 V measure}
%  \label{fig:sfig2}
\end{subfigure}
\caption{S set results}
\label{fig:s_set_results}
\end{figure}

Figure \ref{fig:weighted_issues} shows a snapshot of the Weighted Spectral Clustream algorithm on the S1 data set in the middle of the stream. The grey points are the data points observed so far, the location of the letters represent micro-cluster centres which are labelled with the results of the Weighted Spectral Clustering. This is not a good clustering of the data.  We can see that one cluster (letter N) is dominating and many of the outliers of the other clusters have been represented by the N cluster label. This implies that the affinities between the micro-cluster centres on the outskirts of the cluster C are more similar to the outliers of other clusters (such as cluster I) than to the micro-cluster centres at C. This behaviour is very odd and implies that the affinity matrix may not be as sensible as we intended.

\begin{figure}[h]
  \centering
  \includegraphics[width = 10cm]{weighted_issues_N_crop}
  \caption{Snapshot from Weighted Spectral Clustream on S1}
\label{fig:weighted_issues}
\end{figure}

\begin{figure}[h!]
  \centering
  \begin{subfigure}{0.45\textwidth}
    \centering
    \includegraphics[width = \textwidth, height = \textwidth]{s_set/s_set_1_unweighted_affinity.png}
    \caption{Unweighted affinity matrix}
  \label{fig:unweighted_affinity}
  \end{subfigure}
  \begin{subfigure}{0.45\textwidth}
    \centering
    \includegraphics[width = \textwidth, height = \textwidth]{s_set/s_set_1_weighted_affinity.png}
  \caption{Weighted affinity matrix}
  \label{fig:weighted_affinity}
  \end{subfigure}
    \caption{Affinity matrices for S set 1}
  \label{fig:set_1_affinities}
\end{figure}

We can observe the affinity matrix by plotting it as an image, where bright values imply an affinity value close to 1 and red means the value is close to zero. Figure \ref{fig:set_1_affinities} shows an image of both the unweighted and weighted affinity for the example shown in Figure \ref{fig:weighted_issues}. The affinity matrix has dimension $150 \times 150$, with each row representing the similarities between one micro-cluster centre and the other 149. The rows and columns have been grouped so that micro-cluster centres from the same true underlying clusters are next to each other. 

We can see that in Figure \ref{fig:unweighted_affinity} the block nature of the unweighted affinity is clear. There are strong affinities between close micro-clusters, and weak affinities between distance micro-clusters making this an informative affinity matrix to use with Spectral Clustering. However in the weighted affinity matrix (Figure \ref{fig:weighted_affinity}) the block nature is not visible. Most of the values are close to zero, with only a few strong affinities. The weighting seems to have dampened the affinity matrix, making the clustering problem more difficult.  It is possible that the use of the localised scaling parameter (See Section \ref{sec:affinity}) in the Spectral Clustering step may be interfering with the weighting of micro-clusters. We did try using a global scaling parameter instead of the localised one, however this then brought up the issue of tuning the $\sigma$ parameter, which is known to be very sensitive and is a difficulty for Spectral Clustering algorithms in general \citep{Luxburg2008}. Although performance did seem to improve with the global scaling parameter when chosen carefully, the performance was still very poor compared to Windowed Spectral Clustering and Spectral Clustream.

%From observation we noted that once this behaviour of  an ``absorbing outlier cluster'' began, the algorithm did not recover hence the drop in performance seen in Figure \ref{fig:s_set_results}.

\subsection{Real texture data}

We now investigate the performance of the clustering algorithms on real data. This new data set consists of features extracted from  textured images. The Kylberg  texture data set \citep{kylberg2011c} consists of 28 texture classes with 160 unique texture patches per class. The patches consist of $576 \times 576$ pixel images. Features for clustering were extracted using the LS2W \citep{Eckley2011} which creates 27 wavelet features from the textured images. 

A subset of 6 classes were selected,  examples of which are shown in Figure \ref{fig:texture_examples} .  The classes selected are images of two types of blanket, some canvas, a ceiling, some lentils and a screen.

\begin{figure}[h!]
\begin{subfigure}{.15\textwidth}
  \centering
  \includegraphics[width=.8\linewidth]{kylberg_examples/blanket1_001.png}
  %\caption{PCA of digits 2 and 9}
 % \label{fig:sfig1}
\end{subfigure}%
\begin{subfigure}{.15\textwidth}
  \centering
  \includegraphics[width=.8\linewidth]{kylberg_examples/blanket2_001.png}
  %\caption{Purity digits 2 and 9}
%  \label{fig:sfig2}
\end{subfigure}
\begin{subfigure}{.15\textwidth}
  \centering
  \includegraphics[width=.8\linewidth]{kylberg_examples/canvas1_001.png}
  %\caption{V-measure digits 2 and 9}
%  \label{fig:sfig2}
\end{subfigure}
\begin{subfigure}{.15\textwidth}
  \centering
  \includegraphics[width=.8\linewidth]{kylberg_examples/ceiling1_001.png}
  %\caption{PCA of digits 2 and 9}
 % \label{fig:sfig1}
\end{subfigure}%
\begin{subfigure}{.15\textwidth}
  \centering
  \includegraphics[width=.8\linewidth]{kylberg_examples/lentils1_001.png}
  %\caption{Purity digits 2 and 9}
%  \label{fig:sfig2}
\end{subfigure}
\begin{subfigure}{.15\textwidth}
  \centering
  \includegraphics[width=.8\linewidth]{kylberg_examples/screen1_001.png}
  %\caption{V-measure digits 2 and 9}
%  \label{fig:sfig2}
\end{subfigure}

\begin{subfigure}{.15\textwidth}
  \centering
  \includegraphics[width=.8\linewidth]{kylberg_examples/blanket1_002.png}
  %\caption{PCA of digits 2 and 9}
 % \label{fig:sfig1}
\end{subfigure}%
\begin{subfigure}{.15\textwidth}
  \centering
  \includegraphics[width=.8\linewidth]{kylberg_examples/blanket2_002.png}
  %\caption{Purity digits 2 and 9}
%  \label{fig:sfig2}
\end{subfigure}
\begin{subfigure}{.15\textwidth}
  \centering
  \includegraphics[width=.8\linewidth]{kylberg_examples/canvas1_002.png}
  %\caption{V-measure digits 2 and 9}
%  \label{fig:sfig2}
\end{subfigure}
\begin{subfigure}{.15\textwidth}
  \centering
  \includegraphics[width=.8\linewidth]{kylberg_examples/ceiling1_002.png}
  %\caption{PCA of digits 2 and 9}
 % \label{fig:sfig1}
\end{subfigure}%
\begin{subfigure}{.15\textwidth}
  \centering
  \includegraphics[width=.8\linewidth]{kylberg_examples/lentils1_002.png}
  %\caption{Purity digits 2 and 9}
%  \label{fig:sfig2}
\end{subfigure}
\begin{subfigure}{.15\textwidth}
  \centering
  \includegraphics[width=.8\linewidth]{kylberg_examples/screen1_002.png}
  %\caption{V-measure digits 2 and 9}
%  \label{fig:sfig2}
\end{subfigure}

\begin{subfigure}{.15\textwidth}
  \centering
  \includegraphics[width=.8\linewidth]{kylberg_examples/blanket1_003.png}
  %\caption{PCA of digits 2 and 9}
 % \label{fig:sfig1}
\end{subfigure}%
\begin{subfigure}{.15\textwidth}
  \centering
  \includegraphics[width=.8\linewidth]{kylberg_examples/blanket2_003.png}
  %\caption{Purity digits 2 and 9}
%  \label{fig:sfig2}
\end{subfigure}
\begin{subfigure}{.15\textwidth}
  \centering
  \includegraphics[width=.8\linewidth]{kylberg_examples/canvas1_003.png}
  %\caption{V-measure digits 2 and 9}
%  \label{fig:sfig2}
\end{subfigure}
\begin{subfigure}{.15\textwidth}
  \centering
  \includegraphics[width=.8\linewidth]{kylberg_examples/ceiling1_003.png}
  %\caption{PCA of digits 2 and 9}
 % \label{fig:sfig1}
\end{subfigure}%
\begin{subfigure}{.15\textwidth}
  \centering
  \includegraphics[width=.8\linewidth]{kylberg_examples/lentils1_003.png}
  %\caption{Purity digits 2 and 9}
%  \label{fig:sfig2}
\end{subfigure}
\begin{subfigure}{.15\textwidth}
  \centering
  \includegraphics[width=.8\linewidth]{kylberg_examples/screen1_003.png}
  %\caption{V-measure digits 2 and 9}
%  \label{fig:sfig2}
\end{subfigure}
\caption{Three examples from each of the 6 different texture tiles used in the streaming data set. The texture classes are (L-R) Blanket 1, Blanket 2, Canvas, Ceiling, Lentils and Screen}
\label{fig:texture_examples}
\end{figure}

The performance plots for the texture data are shown in Figure \ref{fig:texture_results}. Here we  do see a difference between Windowed Spectral Clustering and Unweighted Spectral Clustream, the windowed approach is generally performing better. Once again Weighted Spectral Clustream does not perform well, and performance declines as the stream progresses.

\begin{figure}[h!]
\begin{subfigure}{.5\textwidth}
  \centering
  \includegraphics[width=1\linewidth]{texture/texture_with_weighted_ci_one_size_purity}
  \caption{Texture Purity}
 % \label{fig:sfig1}
\end{subfigure}%
\begin{subfigure}{.5\textwidth}
  \centering
  \includegraphics[width=1\linewidth]{texture/texture__with_weighted_ci_one_size_vmeasure}
  \caption{Texture V-measure}
%  \label{fig:sfig2}
\end{subfigure}
\caption{Texture Results}
\label{fig:texture_results}
\end{figure}


The poor performance of Weighted Spectral Clustream was observed on all other data sets investigated and therefore the results from this algorithm  will be dropped from any further performance plots in order to focus more on the behaviour of the other two algorithms.  From now on, we will refer to Unweighted Spectral Clustream simply as Spectral Clustream.

\subsection{Real Pendigit data}
This section and the next will use the UCI Pendigit data set which was introduced in \cite{Alimoglu1996} and is available is for download \citep{Lichman2013}. The data set consists of 250 samples of hand drawn digits of the numbers 0-9 taken from 44 writers. The data was collected using a pressure sensitive tablet. There are 16 features each relating to the co-ordinate information taken from the input tablet. We restrict our analysis to pairwise comparison of digits. For example we attempt to cluster the digits 0 and 1, and treat the data as if it is arriving in a constant data stream. 

The results for a selection of the pairwise digits are shown in Figure \ref{fig:uci_pendigits}. The first column displays the digit data in PCA space, the second column shows the purity, and the third column shows the V-measure. Both algorithms show similar performance in the plots shown (and in all the other pairwise combinations which were run).  It can be noted that V-measure is lower than purity, which might imply that the resulting clusters are homogeneous but not complete (see Section \ref{sec:performance}). 

\begin{figure}[H]

\begin{subfigure}{.3\textwidth}
  \centering
  \includegraphics[width=\linewidth]{PCA_pendigits_pairwise/pairwise_1_6_cropped.png}
  \caption{PCA of digits 1 and 6}
 % \label{fig:sfig1}
\end{subfigure}%
\begin{subfigure}{.3\textwidth}
  \centering
  \includegraphics[width=\linewidth]{pendigits_2_alg/uci_pendigits_16_ci_one_size_purity.png}
  \caption{Purity digits 1 and 6}
%  \label{fig:sfig2}
\end{subfigure}
\begin{subfigure}{.3\textwidth}
  \centering
  \includegraphics[width=\linewidth]{pendigits_2_alg/uci_pendigits_16_ci_one_size_vmeasure.png}
  \caption{V-measure digits 1 and 6}
%  \label{fig:sfig2}
\end{subfigure}

\begin{subfigure}{.3\textwidth}
  \centering
  \includegraphics[width=\linewidth]{PCA_pendigits_pairwise/pairwise_4_7_cropped.png}
  \caption{PCA of digits 4 and 7}
 % \label{fig:sfig1}
\end{subfigure}%
\begin{subfigure}{.3\textwidth}
  \centering
  \includegraphics[width=\linewidth]{pendigits_2_alg/uci_pendigits_47_ci_one_size_purity.png}
  \caption{Purity digits 4 and 7}
%  \label{fig:sfig2}
\end{subfigure}
\begin{subfigure}{.3\textwidth}
  \centering
  \includegraphics[width=\linewidth]{pendigits_2_alg/uci_pendigits_47_ci_one_size_vmeasure.png}
  \caption{V-measure digits 4 and 7}
%  \label{fig:sfig2}
\end{subfigure}
\caption{Pendigits Pairwise - Spectral Clustream and Windowed Spectral Clustering}
\label{fig:uci_pendigits}
\end{figure}

\subsection{Real non-stationary data}

So far in this section we have considered only static data sets. The main challenge for clustering algorithms for data streams is adapting to changes in the data set. In order to create real data sets with a non stationary distribution we introduce a change into the data set. To construct a data set we choose three digits in the Pendigits data set (for example 4, 8 and 9). The start of the data stream consists only of 2 digits (4 and 8). Half way through the stream, we replace  the second digit with the third digits (so now we observe values 4 and 9 rather than 4 and 8). By swapping the digits in this manner we can avoid the difficult issue of having to select a number of clusters, as we always observe only two clusters. 

Due to the nature of algorithms with windowing, we would expect Windowed Spectral Clustering to adapt quickly to this change given the window size is fairly small ($w = 150$). The initial results showed Spectral Clustream performing poorly after the change, and struggling to recover. These results will be shown at the end of the section, and fix is suggested to correct the poor behaviour. First, in order to understand why Spectral Clustream was performing badly lets consider a specific example. 
The example set is ``Pendigits 48 49'' - first we observe features from digits 4 and 8, and we switch to observing features from digits 4 and 9 half way through the stream. Figure \ref{fig:48_49_pca} shows the PCA space for all  three digits. There is quite a bit of overlap between digits 4 and 9, which makes the set quite tricky to cluster. 

\begin{figure}[!h]
  \centering
 \includegraphics[width=.6\linewidth]{evolving_pen/evolving_pen_48_49_truth.png}  
\caption{PCA plot for the Pendigits 4, 8 and 9}
\label{fig:48_49_pca}
\end{figure}

Figure \ref{fig:stream_1} shows the Spectral Clustream at the start of the stream (directly after initialisation). The grey points show all data seen up until this time and the blue points indicate the next 200 points to be observed (the test set used for our performance measures.) The crosses indicate locations of micro-cluster centres, and their colour indicates which overall cluster they have been assigned to using the Spectral Clustering algorithm. 


\begin{figure}
\begin{subfigure}{.5\textwidth}
  \centering
  \includegraphics[width=\textwidth]{evolving_pen/evolving_pen_48_49_1_crop.png}  
  \caption{Stream at time t = 1}
  \label{fig:stream_1}
\end{subfigure}
\begin{subfigure}{.5\textwidth}
  \centering
  \includegraphics[width=\textwidth]{evolving_pen/evolving_pen_48_49_100_crop.png}  
  \caption{Stream at time t = 1000 }
  \label{fig:stream_1000}
\end{subfigure}
\caption{Micro-cluster centres for the Pendigits 48, 49}
\label{stream_1_100}
\end{figure}

In Figure \ref{fig:stream_1}, at time step 1, the micro-cluster centres are well distributed over the grey data points and also the blue test points. The Spectral Clustering mostly segments the micro-cluster centres correctly into the left and right clusters.

At time step 1000 (\ref{fig:stream_1000}), we have begun observing the new cluster (digit 9 in the top left corner) and are no longer receiving data about digit 8. This is shown as all of the test data points (light blue) are in either the bottom left corner (digit 4) and top left corner (digit 9). We see that the micro-cluster centres (crosses) are spread out over all the data including the defunct cluster of digit 8 (the grey data points on the right hand side of the plot).  The reason that there are still micro-cluster centres located in the cluster 8 region is because of the deletion policy that Clustream uses.

Clustream requires the number of micro-clusters to remain fixed for the duration of the stream. As discussed in Section \ref{sec:microSpec}, if an arriving data point does not have a suitable micro-cluster to merge into, a new micro-cluster is formed. However, since the total number of micro-clusters is fixed, this means that we need to either delete an old micro-cluster if it is suitably old, or combine two close ones. This is the only way that micro-clusters can be deleted in Clustream.  A micro-cluster $i$ is defined to be suitably old if it's relevancy $r(M_i)$ is less than the relevancy threshold $\delta$, as defined and discussed in equation \eqref{eq:clustream_deletion}.

In practice it is difficult to select the best value of $\delta$. If $\delta$ is set too high, Clustream will delete too often and therefore emerging new micro-clusters may not be allowed to develop fully. If $\delta$ is set too small then old clusters will stay in the system much longer than required. This is an example of when $\delta$ is possibly too small, as the algorithm seems unwilling to delete old micro-clusters. %One way to improve the algorithm would be to change the Clustream deletion policy to something more intuitive, perhaps by removing the assumption that the time stamps are normally distributed. However, due to Clustream's success I wish to keep the algorithm as close to the original as possible. 

In itself having old micro-cluster centres remain isn't a problem. Keeping old micro-cluster centres is a useful way to retain some historical data about the data stream. Also, in the case where a cluster disappears for some time and then re-emerges later in the stream, retaining old micro-cluster centres may speed up the learning when the old cluster re-emerges. These type of scenarios can occur regularly in any sort of cyclic data, such as any shopping data which has seasonality. 

The problem arises when the old micro-cluster centres are used in the Spectral Clustering step. By including these old centres in the Spectral Clustering, the algorithm technically has centres from three clusters (digits 4, 8 and 9), but we have asked the Spectral Clustering algorithm to find two clusters. In the example above in Figure \ref{fig:stream_1000}, we see that the two clusters on the left that we are trying to separate are grouped together as one, because of the inclusion of the old micro-cluster centres on the right of the plot.

The proposed solution to deal with these micro-cluster centres is as follows. Before the macro-clustering step is complete, identify the micro-clusters of interest. In this setting, we find the micro-clusters which the test data are closest to. Then use only these relevant micro-cluster centres to perform the Spectral Clustering. Figure \ref{fig:stream_standard_alt} compares the  previous method with the proposed alteration. Figure \ref{fig:stream_1_repeat} shows the standard algorithm at t = 1000 (this is a duplicate of Figure \ref{fig:stream_1000} repeated for comparison purposes). Figure \ref{fig:stream_100_alt} shows the clustering of the micro-cluster centres when the  Alternative Spectral Clustream is used. 

\begin{figure}
\begin{subfigure}{.5\textwidth}
  \centering
  \includegraphics[width=\textwidth]{evolving_pen/evolving_pen_48_49_100_crop.png}  
  \caption{Standard Spectral Clustream Algorithm}
  \label{fig:stream_1_repeat}
\end{subfigure}
\begin{subfigure}{.5\textwidth}
  \centering
 \includegraphics[width=\textwidth]{evolving_pen/alternative_evolving_pen_48_49_100_crop.png}  
  \caption{Alternative Spectral Clustream Algorithm}
  \label{fig:stream_100_alt}
\end{subfigure}
\caption{Micro-cluster centres for the Evolving Pendigits 48, 49 at t = 1000}
\label{fig:stream_standard_alt}
\end{figure}


We can see that the older micro-cluster centres are no longer used in the Spectral Clustering, and therefore the algorithm is better able to distinguish between the digit 4 and the digit 9. Note that although the old micro-cluster centres are not shown in the figure, they are technically still stored, but since they are not used in the macro-clustering stage they do not receive a cluster label therefore are not shown on the plot. 

Once obvious issue with this amendment is there may be fewer input data points into the Spectral Clustering algorithm - this means that the number of micro-clusters used becomes more important.  In the toy example above we used 50 micro-clusters to represent the stream. This is already fairly small, but given that some centres are now not used in the Spectral Clustering step, this can be an issue. In fact at time step 1000 (Figure \ref{fig:stream_100_alt}), we can see that only 19 of the 50 micro-clusters are being used in the Spectral Clustering algorithm. Therefore there may be a need when using this alternative method to select the number of micro-clusters to be larger then required.  

Figures \ref{fig:standard_alternative_4849} and \ref{fig:standard_alternative_3437} show the alternative Spectral Clustream performance with the standard Spectral Clustream performance. We have included a number of different values for the number of micro-clusters  $(50, 100, 150, 200)$ and also ran Windowed Spectral with window sizes $w = 50,100,150, 200$.  In the plots the dashed lines show the Windowed Spectral Clustering and the full lines are Unweighted Spectral Clustream. The colours indicate different values for the number of micro-clusters/window size. The plots on the left show the performance for the Standard Unweighted Spectral Clustering, and the plots on the right show the proposed Alternative Unweighted Spectral Clustering. The performance of Windowed Spectral Clustering is repeated in both left and right plots for comparison purposes. 

%%%%%%%%%%%%%%%%%%%%%%%%%%%%%%%%%%%%%%%%%%%%%%%%%%%%%%%%%%%%%%%%%%%%%%%%%%%%%%%%%%%%%%%%%%%%%%%%%%%%%%%%%%%%%%%%%%%%%
%48 - 49
%%%%%%%%%%%%%%%%%%%%%%%%%%%%%%%%%%%%%%%%%%%%%%%%%%%%%%%%%%%%%%%%%%%%%%%%%%%%%%%%%%%%%%%%%%%%%%%%%%%%%%%%%%%%%%%%%%%%%
\begin{figure}[!h]
        \centering
        \begin{subfigure}[b]{0.47\textwidth}
          \includegraphics[width=\textwidth]{standard_alt/ci_evolving_pen_48_49_standard_purity.png}         
                 \caption{Purity - Standard}
                 \label{fig:ps_4849}
        \end{subfigure}
        \begin{subfigure}[b]{0.47\textwidth}
                 \includegraphics[width=\textwidth]{standard_alt/ci_evolving_pen_48_49_alternative_purity.png}
                \caption{Purity - Alternative}
                \label{fig:pa_4849}
        \end{subfigure}
%\end{figure}
%\begin{figure}[h]
%        \centering
        \begin{subfigure}[b]{0.47\textwidth}
          \includegraphics[width=\textwidth]{standard_alt/ci_evolving_pen_48_49_standard_vmeasure.png}
                 \caption{Vmeasure - Standard}
                 \label{fig:vs_4849}
        \end{subfigure}
        \begin{subfigure}[b]{0.47\textwidth}
                 \includegraphics[width=\textwidth]{standard_alt/ci_evolving_pen_48_49_alternative_vmeasure.png}
                \caption{Vmeasure - Alternative}
                \label{fig:va_4849}
        \end{subfigure}
\caption{Standard vs Alternative Clustream on Pendigits 48 49}
\label{fig:standard_alternative_4849}
\end{figure}

%%%%%%%%%%%%%%%%%%%%%%%%%%%%%%%%%%%%%%%%%%%%%%%%%%%%%%%%%%%%%%%%%%%%%%%%%%%%%%%%%%%%%%%%%%%%%%%%%%%%%%%%%%%%%%%%%%%%%
%34 - 37
%%%%%%%%%%%%%%%%%%%%%%%%%%%%%%%%%%%%%%%%%%%%%%%%%%%%%%%%%%%%%%%%%%%%%%%%%%%%%%%%%%%%%%%%%%%%%%%%%%%%%%%%%%%%%%%%%%%%%
\begin{figure}[!h]
        \centering
        \begin{subfigure}[b]{0.47\textwidth}
          \includegraphics[width=\textwidth]{standard_alt/ci_evolving_pen_34_37_standard_purity.png}         
                 \caption{Purity - Standard}
                 \label{fig:ps_3437}
        \end{subfigure}
        \begin{subfigure}[b]{0.47\textwidth}
                 \includegraphics[width=\textwidth]{standard_alt/ci_evolving_pen_34_37_alternative_purity.png}
                \caption{Purity - Alternative}
                \label{fig:pa_3437}
        \end{subfigure}
%\end{figure}
%%%%%%%%%%%%%%%%%%%%
%\begin{figure}[h]
%        \centering
        \begin{subfigure}[b]{0.47\textwidth}
          \includegraphics[width=\textwidth]{standard_alt/ci_evolving_pen_34_37_standard_vmeasure.png}
                 \caption{Vmeasure - Standard}
                 \label{fig:vs_3437}
        \end{subfigure}
        \begin{subfigure}[b]{0.47\textwidth}
                 \includegraphics[width=\textwidth]{standard_alt/ci_evolving_pen_34_37_alternative_vmeasure.png}
                \caption{Vmeasure - Alternative}
                \label{fig:va_3437}
        \end{subfigure}
\caption{Standard vs Alternative Clustream on Pendigits 34 37}
\label{fig:standard_alternative_3437}
\end{figure}

The plots show that although Unweighted Spectral Clustream was performing poorly when using the standard approach, using the alternative method has brought performance back up to being as good as Windowed Spectral Clustering.  The importance of the number of micro-clusters is evident in this change set.  In Figure \ref{fig:va_4849} we can clearly see the number of micro-clusters affects the performance of the algorithm. Initially the number of micro-clusters doesn't have a large effect on performance but after the change occurs, using 50 micro-clusters was not sufficient for the algorithm to adapt. However using 200 micro-clusters brought performance up to a reasonable level.  We also observe the effect of window size on performance. In Figure \ref{fig:pa_3437} it is clear at the point of change that the window with the smallest size (50) recovers first, with the largest window size taking the longest to adapt to the change.

\section{Conclusion and Future Work}
\label{sec:clustream_conc}

In this Chapter we have presented Spectral Clustream, a clustering method capable of performing Spectral Clustering on data streams. Under the suggestion from \cite{Zhang1996a}, we considered two variations of this algorithm, a weighted and an unweighted algorithm. Despite having a mathematically valid affinity matrix, the Weighted Spectral Clustream was found to have very poor performance and fundamental difficulties clustering even simple simulated data. The Unweighted Spectral Clustream was shown to have good performance on par with a windowed approach to Spectral Clustering. An issue with the Unweighted Spectral Clustream deletion policy was highlighted, where we saw historic micro-clusters being retained causing to the Spectral Clustering to fail. A fix was suggested in order to retain the good performance, but at the cost of using additional micro-clusters to track the stream.

There is still much interesting work to be done in this area particularly on why Weighted Spectral Clustream does not perform well empirically. Further investigation into the effect of the scaling parameter $\sigma$ and whether localisation affects performance negatively is required. The deletion policy of Clustream also has issues which need addressing, starting perhaps with removing the assumption that time stamps arrivals are Normally distributed. It is common to assume that the true number of clusters is known, as we have done in this Chapter. However, in practice this is not generally the case. Choosing the number of underlying clusters in a streaming setting is an interesting challenge for future research.

There does not exist an obvious method for evaluating performance of clustering algorithms in the streaming setting. The current standard performance measures take into account only how well the current clustering represents the current state of the stream. It would be useful to develop methods appropriate for evaluating non-stationary data streams. For example, we might wish to judge how well a clustering algorithm learns about the historical state of the stream as well its ability to adapt quickly to new information.
