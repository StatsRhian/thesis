\documentclass{article}

\begin{document}

\section{Notes on Laplacians}

There are three major Graph Laplacians.

Don't assume eigenvectors of a matrix are normalised. (Normalised means the eigenvector has length one).

Assume eigenvectors ordered increasing (first $k$ eigenvectors means the $k$ smallest eigenvectors).


Unnormalized Laplacian is L = D - W

Normalized Laplacians

$L_{sym} := D^{-1/2}LD^{-1/2} = I - D^{-1/2}WD^{-1/2}$

This second bit follows since $D^{-1/2} D D^{-1/2}$

Note: How to calculate $D^{1/2}$
Let $V$ be the vector of eigenvalues of $D$, and let $Q$ be a matrix of the eigenvectors of $D$. 
Then $D^{1/2} = Q diag(\frac{1}{\sqrt{V}})Q^T$

Note that $L_{symm}$ is symmetric. 

The other normalized Laplacian commonly used is $L_{rw} = D^{-1}L = I - D^{-1}W$

Reference Chung 1997

Dave says that the smallest eigenvectors of $D^{-1/2}L_{symm}D^{-1/2}$ are the same as the largest eigenvectors of $D^{-1/2}WD^{-1/2}$.

This makes sense! As the eigenvalues of the identity matrix are all one. 

Let $e1$ be the eigenvalues of $D^{-1/2}WD^{-1/2}$. and $e2$ be the eigenvalues of $D^{-1/2}LD^{-1/2}  = L_{symm}$, 
then we see that $e2 = 1 - e1$ but in the reverse order. Therefore the e2 is increasing in size and e1 is decreasing in size. 

Lets now look at incorporating an adapted affinty matrix.

According to Dave...

Let $s = (s_1, s_2, \hdots, s_k)$ be a vector where $s_i$ represents the number of points assigned to the $i^{th}$ microcluster.

We wish to have an Affinity weighted by the 

Let $A \in \mathbb{R}^{n \times n}$ be the affinity matrix of the micro-clustered data, i.e., with repeated elements for the data in each microcluster. Assume that the columns (and therefore rows) of A are ordered such that the first $n_1$ are associated with the data assigned to microcluster 1, which has size $n_1$ and the next $n_2$ with those assigned to microcluster 2, and so on.

Let $\tilde{A}^{\prime} \in \mathbb{R}^{k \times k}$ have $i,j$-th element equal to $n_in_jA^{\prime}_{ij}$. $\tilde{A}$ is a valid affinity matrix since it is symmetric with non-negative entries. 
If we wish to have $\tilde{A}_{ij} \leq 1$ then simply divide $\tilde{A}$ by $max_i n_i^2$, but this makes no difference to the spectral decomposition. 

Let $D, D^{\prime}, \tilde{D}$ be the corresponding degree matricies and $L,  L^{\prime}, \tilde{L}$ be the corresponding normalized symmetric Laplacians.

-----


Rcode

Firstly for the normalized Symmetric Laplacian.

Choose $\tilde{A} = s%*%t(s)*A^{prime}$

\end{document}
