%---------------- COMMENT FOR IMPORTING ----------------------
%\RequirePackage{lineno}					%Comment for importing
%\documentclass[12pt]{report}		%Comment for importing
%\pagestyle{headings}
%\input{ST-input2}								%Comment for importing

%\setcounter{chapter}{2}
%\begin{document}								%Comment for importing
%\setpagewiselinenumbers
%\linenumbers
%\tableofcontents
%-------------------------------------------------------------

\chapter{Conclusions and Future Work}
\label{chap:conc}
\begin{itemize}
\item Summarise all contributions
\item Online EM links with Clustream
\item Difficulty in evaluating performance on data streams
\end{itemize}


Notes from Dave's Thesis 1.3

The practical challenges associated with clustering data streams are clear,
however the data stream paradigm also poses philosophical questions about
what defines a cluster. Consider a situation in which the generative process
undergoes an abrupt change, such that the data distribution after the change
is essentially unrecognisable in the context before the change. Even if the
number of clusters under the new distribution is the same, how can one associate
data which arrived before the change with those arriving afterwards?
If the location of such abrupt changes is known, it could be argued that
the previous clusters no longer exist, and any information drawn from the
new clusters should be independent of what came before. However, abrupt
changes might not affect all clusters, and discarding past information could
be detrimental if unnecessary. If the location of changes is unknown, then
the problem becomes immeasurably more complex. Attempting to describe
clusters as persistent entities is almost paradoxical if clusters are described as
groups of data. It seems necessary, therefore, to attempt to estimate features
of the underlying generative process and define clusters relative to them. In
addition, it is preferable to identify changes in the generative process which
affect the cluster definitions, rather than discounting information from previous
data which may or may not be related to the data relevant to the current
cluster definitions. Ideally, in addition the identification of changes to the
process should be at a local level so that information from persistent features
of the process is not discarded.







%---------------- COMMENT FOR IMPORTING ----------------------
%\pagebreak											%Comment for importing
%\bibliographystyle{plainnat}		%Comment for importing
%\bibliography{References}				%Comment for importing
%\end{document}									%Comment for importing
%-------------------------------------------------------------

